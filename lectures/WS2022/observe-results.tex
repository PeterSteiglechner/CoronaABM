% !TeX root = CoronaABM.tex

%\begin{frame}
%	\frametitle{Observe Function}
%	Don't show code in detail, just results \ra next slides.\\
%	Show how to get these aggregate results (list comprehension and ask for state, then count states)
%\end{frame}

\begin{frame}[fragile]
	\frametitle{Run and Observe Function}

Goal: Perform simulation and track how many agents are in state "susceptible", "exposed", ... over time.
\begin{overlayarea}{\linewidth}{12cm}
\begin{onlyenv}<1>
\begin{lstlisting}[basicstyle=\tiny\ttfamily, tabsize=6] 
initialise() 
network = initialise_network(agents, k_ws=6, p_ws=0.1)

T_ARRAY = np.linspace(0, 400, 0.5)

results = np.empty([len(T_ARRAY), len(states)])

for n, t in enumerate(T_ARRAY):
      update(t)
      results[n, :] = observe(agents)
\end{lstlisting}
        
\begin{lstlisting}[basicstyle=\tiny\ttfamily, tabsize=6] 
      
def observe(agents):
      states_of_agents = [ag.state for ag in agents]
      N_s = states_of_agents.count("susceptible")
      N_e = states_of_agents.count("exposed")
      ...
      return np.array([N_s, N_e, N_ia, ...])
\end{lstlisting}
\end{onlyenv}
\begin{onlyenv}<2>
\begin{figure}
	\centering
	\includegraphics[width=0.8\linewidth]{outputimages_2022/AggregateResults_N-1000-2_WS(6,1e-01)_seed11}

        {\small \phantom{Another\ }Outbreak in a network with $k=6$, $p = 0.2$, $n=1000$ agents of which two are exposed at $t=0$.}
	\label{fig:aggregateresultsn-1000-2ws62e-01seed1}
\end{figure}
\end{onlyenv}
\begin{onlyenv}<3>
	\begin{figure}
		\centering
	\includegraphics[width=0.8\linewidth]{outputimages_2022/AggregateResults_N-1000-2_WS(6,1e-01)_seed112}

                {\small \textit{Another} outbreak in a network with $k=6$, $p = 0.2$, $n=1000$ agents of which two are exposed at $t=0$.}
		\label{fig:aggregateresultsn-1000-2ws62e-01seed1}
	\end{figure}
\end{onlyenv}
\begin{onlyenv}<4>
\begin{figure}
	\centering
	\includegraphics[width=0.8\linewidth]{outputimages_2022/AggregateResults_N-1000-2_WS(6,1e-01)_seed1}

        {\small The model is {\color{red}stochastic}! \newline Same model configuration \ra might not lead to an outbreak at all. Model output depends on the occurrence of microscopic events.}
\end{figure}
\end{onlyenv}
\end{overlayarea}
\end{frame}

\againframe<6>{overview}

\begin{frame}
	\frametitle{Validation}
How can we verify if these results make sense? E.g.\ 
\begin{enumerate} 
\item \textbf{Ensemble runs}: Do many different stochastic simulation runs.
\item \textbf{Compare with data}: For ABM, typically aggregate  indicators.\newline For example, $R_0$ in the data vs.\ mean $R_0$ in ensemble runs of the model.
\end{enumerate}
\pause%\begin{onlyenv}<2>
\includegraphics[width=0.7\linewidth]{outputimages_2022/R0values_N-1000-2_WS(6,1e-01)_seed112}
%\end{onlyenv}

\end{frame}

\section{Policies}

\begin{frame}
  \frametitle{Policies}
	The major idea of our model was to design and test the impact of local policies. \\
	\vspace{0.2cm}

        What happens, for example, 
	\begin{itemize} 
	\item if people reduce their contacts (due to social distancing policies)?
	\item if people keep their contacts within confined clusters (households, neighbours)?
	\item if people reduce their infectiousness by wearing a mask?	
	\end{itemize} 
	
\end{frame}

\begin{frame}
  \frametitle{Parameters of the Model}
\begin{table}
	\scriptsize
\begin{tabularx}{\textwidth}{>{\tiny}p{2.5cm}|X|p{2cm}}
	Parameter & Description & current value \\ \hline
	FRAC\_AGE\_GROUPS & Percentage of each age group & [0.2, 0.5, 0.3]  \\
	BASE\_I & Distribution of base infectiousness of the agents & beta(1, 3) \\
	P\_SYMPTOMATIC & Probability to develop symptoms  & [0.1, 0.5, 0.8] \\
	INCUBATION\_PERIOD & Distribution of the incubation time & gamma(5.807, 0.948)\\
	CFR & Case fatality ratio for each age group & [0.0001, 0.005, 0.05] \\
	RELATIVE\_INFECT-IOUSNESS & rel.\ strength of infectiousness for a (pre-)symptomatic and asymptomatic infections & [0.5,1,0.2] \\   
	TIME\_I\_PRESYMPT & infectious days before symptom onset & 2  \\
	TIME\_I\_POSTSYMPT & infectious days after symptom onset & 4  \\
	k\_ws & Number of (nearest) neighbours in networks for nodes &  6 \\
	p\_ws & Probability of each link to be rewired &  0.1  \\
	N\_AGENTS & Nr of agents & 1000 \\ %  # Number of agents in total
	N\_INFECTED\_INIT & Nr of agents set to "exposed" at t=0 & 2  \\
\end{tabularx}
\end{table}

\end{frame}

\begin{frame}
  \frametitle{Policy: Decrease $k=6$ of network to $k=4$}
\begin{figure}
  \only<1>{\includegraphics[width=\linewidth]{outputimages_2022/AggregateResults_N-1000-2_WS(6,1e-01)_seed4}}%
  \only<2>{\includegraphics[width=\linewidth]{outputimages_2022/AggregateResults_N-1000-2_WS(4,1e-01)_seed4}}
			%\Put(-40,80){\includegraphics[width=0.7\linewidth]{../../../CoronaABM/Simplified_Corona/Figures/R0values_seed2}}}
	\end{figure}
\end{frame}


%%% Local Variables:
%%% mode: latex
%%% TeX-master: t
%%% TeX-master: "CoronaABM"
%%% TeX-master: t
%%% TeX-master: t
%%% End:
