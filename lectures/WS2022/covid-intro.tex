\begin{frame}
  \frametitle{Coronavirus and Covid-19}
  \begin{itemize}
  \item SARS-CoV-2 appears in Wuhan, China (Dec 19)
  \item Spreads mainly airborne (aerosols, ...)
  \item Spreads fast in our societies: \newline
    {\small basic reproductive number $R_0$: `expected number of cases directly generated by one case in a population where all individuals are susceptible to infection'}
  \item Virus \ra disease Covid-19
  \item WHO declares pandemic on 31st Jan 2020 
  \item Cases across the world within three months \pause
  \end{itemize}
  \vfill
  \begin{itemize}
  \item Early 2020: ``How dangerous is Covid-19?''
    \begin{itemize}
    \item High death rates (especially March 2020) % (and case fatality rate)
    \item Fast spreading
    \end{itemize}
  \end{itemize}
  \vfill
  {\tiny Sources e.g.\
    \url{https://www.ecdc.europa.eu/en/covid-19/facts/questions-answers-basic-facts}
  }	
\end{frame}

      
\begin{frame}.
  \frametitle{What made Coronavirus so difficult to manage?}

    \begin{framed}
      {\color{red}\textbf{Your answers}}
    \end{framed}
  
  % COMPLEXITY IN SPREADING
  % uncertainty heterogeneity (some multi-spreader) complexity of spreading
  % DISEASES IS DANGEROUS
  % high infectiousness infectiousness before symptoms?
  % delay between infectiousness and symptoms no protection heterogeneity in danger
  % LITTLE EXPERIENCE
  % no policies no comparison no time
  \vfill
  \pause
  \begin{enumerate}
    \item Complex spreading patterns
    \item Aggressive virus
    \item Specific groups of people especially vulnerable to the diseases
    \item Little prior experience in (political) decision-making
    \end{enumerate}

    \ra In sum: a lot of uncertainty!

    {\scriptsize E.g.\ \citet{he_temporal_2020}}
\end{frame}
        
    
 
%\begin{frame}{Covid-19: Some stats}
%	\begin{itemize}
%		\item Recovery or Death \newline
%		{\color{red}{The Case Fatality Rate $cfr = \frac{\#\ \text{deaths}}{\# \ \text{confirmed cases}}$ depends strongly on age. 
%		\newline 
%		Note, this is not "how deadly the virus is" (\ra Infection Fatality Ratio $ifr =\frac{\#\ \text{deaths}}{\# \ \text{infected cases}}$) but is often used as a proxy.}}
%		\item $R_{0}$-Value \newline
%		{\color{red}{How many people catch the virus from one infected person (on average) in a \textbf{fully susceptible society}. Observed $R_0$-values ranged between $1.5 - 2.5$. However, it is assumed that infectiousness varies strongly among different people$^{\rm 1}$.}}
%		\item Serial Interval \newline
%		{\color{red}{The time between symptom onsets of consecutive infections (person A infected person B): $t_{\rm Symptom, \ B} - t_{\rm Symptom, \ A} \approx 5.8 \ {\rm days}$ \citep{he_temporal_2020}}}
%	\end{itemize}
%{\tiny (1): Adam, D.\ (2020) \href{https://doi.org/10.1038/d41586-020-02009-w}{“A Guide to R — the Pandemic’s Misunderstood Metric.”} in \textit{Nature}. }
%\end{frame}

\begin{frame}
  \frametitle{Policy approaches -- 2020}
  \begin{itemize}
  \item (Nearly) complete lock-down (Italy, 2020)
  \item Quarantine positive cases and first contacts (Germany, 2020)
  \item Homeschooling/-office, social distancing (Germany, 2020)
  \item Individual responsibility to distance (Sweden, 2020)
  \item Herd immunity (?)
  \item `Circuit breaker' (two-week lock-down) (Germany, winter 2020)
  \end{itemize}

  \vspace{0.2cm} \pause
        
  In general: \textbf{social distancing}
  \begin{itemize}
  \item `Social': fewer contacts
    \begin{itemize}
    \item Close schools (asymptomatic infections)
    \item Ban large gatherings
    \end{itemize}
  \item `Distancing': reduced `contact' 
    \begin{itemize}
      \item Avoid close physical contact
      \item Wear a mask
    \end{itemize}
  \end{itemize}
          
  \vfill {\color{red}Policies have very different epidemiological and socio-economic implications}
\end{frame}


\begin{frame}
  \frametitle{What to do?}

  \centering
  \includegraphics[width=0.9\textwidth]{images/news}

  \vspace{0.5cm} \ra A model to the rescue!?
\end{frame}

      

\begin{frame}
  \frametitle{Role of Modeling}
  \begin{overlayarea}{\textheight}{\textwidth}
    \begin{itemize} 
    \item<1-> Initially: basic models to understand epidemiology \newline
      e.g.\ project cases based on reproductive number $R_0$ \newline
      \ra `flatten the curve'. \\[0.5cm]
        \only<2>{\includegraphics[height=0.65\textheight]{images/nytmodel}}
	\item<3-> Later: more complex, realistic models
          \newline
          e.g.\ age-structured, local resolution, realistic human behaviour
          \newline
          \ra what measures work best to reduce pressure on the health system?
          \vspace{0.2cm}
        \item<4-> Models to design and assess policies:
          \begin{itemize}
            \item<4-> Suggest `new' policies that are predicted to be (more) effective \citep{keeling_precautionary_2020}
            \item<4-> Hindsight analysis: which policy strategy was efficient? \citep[e.g.][]{reiner_modeling_2020} (models answer what-if questions)
         \end{itemize}
       \end{itemize}
      \end{overlayarea}
\end{frame}

\begin{frame}
  \frametitle{Role of Modeling -  An ABM shapes national politics}
  \begin{columns}
    \begin{column}[b]{0.5\linewidth}
  \includegraphics[width=0.9\linewidth]{images/britishGov_corona} \newline{ \small \href{https://www.theguardian.com/world/2020/mar/16/new-data-new-policy-why-uks-coronavirus-strategy-has-changed}{The Guardian, 16 Mar 2020}}
\end{column}
\begin{column}[b]{0.5\linewidth}
                \includegraphics[width=0.9\linewidth]{images/covid19} \newline
                { \small  \href{https://www.washingtonpost.com/world/europe/a-chilling-scientific-paper-helped-upend-us-and-uk-coronavirus-strategies/2020/03/17/aaa84116-6851-11ea-b199-3a9799c54512_story.html}{Washington Post, 17 Mar 2020}}
		\end{column}
              \end{columns}
              \vspace{0.5cm} 
                  \begin{itemize}
                  \item March 2020: UK changes its strategy
                  \item Mainly due to one ABM developed at Imperial College \cite{ferguson_report_2020}!
                  \end{itemize}
              \end{frame}
      
      \begin{frame}
        \frametitle{RECAP: Different types of Models}
	\begin{itemize}
	\item ODE-based SIR Models
          \begin{itemize}
          \item Aggregate variables: size of the population of
            susceptible/(exposed)/infected/recovered 
          \item Equations determine how populations co-evolve.
                 
          \end{itemize}
	\item Agent Based Model:
          \begin{itemize}
          \item Discrete entities `agents'
          \item Individual behaviour specified by rules/heuristics (e.g.\ what does an adult do when infected?) \newline \ra produces individual trajectories
          \end{itemize}
	\end{itemize}
	\vfill
	\begin{framed}
          {\color{red}What are advantages/disadvantages of both
            approaches in general and in particular related to Covid-19?}
	\end{framed}
      \end{frame}

      % "Imagine I'm a mayor of a city/the WHO. What model is more appropriate for my purposes? (And for what?)."
      % ADVANTAGES ODE:    simple, large-scale (national), find the big handle
      % ADVANTAGES ABM:    capture complex events/heterogeneities (not every district is the same), closer to the individual (behaviour), test policies explicitly.


      \begin{frame}
        \frametitle{Why is ABM useful in this pandemic}
        {\small 
	\begin{itemize}
		\item The virus and humans act non-homogeneously:
		{
		\begin{itemize}
			\item {\footnotesize \color{gray} The pandemic is not homogeneous in space}
			 \item {\footnotesize \color{gray} Each person has a different health response to catching the virus }
			\item {\footnotesize \color{gray} Each person behaves differently: e.g.\ number of friends, degree of compliance with policies
			}
			\end{itemize} 
		}
		\item The modeled world is discrete and we need to simulate the actual sequence of events.  \newline
                       {\footnotesize \color{gray} If we had two entirely segregated societies, an outbreak in society A would not impact society B at all.}
		\item Single, random and `microscopic' events are crucial! Reality is path-dependent! \newline
		{\footnotesize \color{gray}
			E.g.\ carnival in Heinsberg (Germany). The pandemic wouldn't exist if "patient 0" in Wuhan had somehow avoided all contacts while infectious
		}
		\item Uncertainty is crucial: \newline
		{
			\footnotesize \color{gray} We cannot fully predict the dynamics.
		}
		\item The course of small-scale events/decisions changes the `rules of the game' (non-ergodic system) \newline
		{
				\footnotesize \color{gray} Wuhan citizens react differently to a 2nd wave than Bremen citizens. 
			}
                      \end{itemize}
                      % \vfill {\small for details: \citet{bookstaber_end_2017,}}
          }

\end{frame}

%%% Local Variables:
%%% mode: latex
%%% TeX-master: "CoronaABM"
%%% End:
