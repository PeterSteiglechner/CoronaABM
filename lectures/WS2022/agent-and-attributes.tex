     
\begin{frame}
  \frametitle{What do we need to model Covid-19}
        \begin{itemize}
        \item Understand individual infection course \newline  {\color{gray} what happens with infected people} %\visible<2>{{\color{red} \ra Were you infected? Recall your infection course?}}
        \item Understand transmissions \newline {\color{gray} how do people infect each other}
        \item Understand social structure \newline  {\color{gray} who infects who}
        \end{itemize}
      \end{frame}

      \begin{frame}
        \frametitle{Covid-19: Course of an infection (2020)}
        \begin{overlayarea}{0.9\textwidth}{0.8\textheight}
\only<1->{Latent period? Pre-Symptomatic and symptomatic, or asymptomatic infection?}%
\vspace{0.4cm}
\begin{center}
	\only<2>{\includegraphics[width=\textwidth]{images/courseInfection_Corona1}} %
		\only<3>{\includegraphics[width=\textwidth]{images/courseInfection_Corona2}} %
		\only<4>{\includegraphics[width=\textwidth]{images/courseInfection_Corona3}} %
                \only<5->{\includegraphics[width=\textwidth]{images/courseInfection_Corona4}} %
              \end{center}
                \vfill
                \vspace{0.4cm}
                \only<6->{\ra infection course is age-dependent}
		%{\tiny \color{gray}  Meyerowitz, EA, et al.\ (2020) \href{https://doi.org/10.1096/fj.202000919}{Rethinking the role of hydroxychloroquine in the treatment of COVID‐19}. The FASEB Journal.}
              \end{overlayarea}
\end{frame}

\begin{frame}
  \frametitle{Model sketch}
  \begin{columns}
    \begin{column}{0.75\textwidth}
  \centering
  \only<1>{\includegraphics[height=0.5\textheight]{images/model0}}%
  \only<2>{\includegraphics[height=0.5\textheight]{images/model1}}%
  \only<3>{\includegraphics[height=0.5\textheight]{images/model2}}%
  \only<4>{\includegraphics[height=0.5\textheight]{images/model3}}%
  \only<5>{\includegraphics[height=0.5\textheight]{images/model4}}%
  \only<6>{\includegraphics[height=0.5\textheight]{images/model0}}%

  \vfill
  \only<1->{\includegraphics[width=0.8\textwidth]{images/courseInfection_Corona4}} %
\end{column}
\begin{column}{0.2\textwidth}
\only<2->{  simple contagion \newline \ \newline  vs. \newline \  \newline complex contagion}
  
  \end{column}
\end{columns}

\end{frame}


\begin{frame}<1-2>[label=Init, fragile]
\frametitle{Agent and initialisation}
\begin{overlayarea}{\linewidth}{12cm}
\begin{onlyenv}<1>
\begin{lstlisting}[basicstyle=\tiny\ttfamily, tabsize=6] 


	
	
class Agent:
	pass
\end{lstlisting}

  \vspace{0.2cm}

An empty class.
\end{onlyenv}
\begin{onlyenv}<2>
\begin{lstlisting}[basicstyle=\tiny\ttfamily, tabsize=6] 
N_AGENTS = 1000 



class Agent:
	pass

def initialise():
	global agents
	agents = []

	for i in range(N_AGENTS):
		ag = Agent()
		ag.id = i
		ag.health_state = "susceptible"	
		ag.age = ???			
		agents.append(ag)
\end{lstlisting} 
  \vspace{0.2cm}

  Create all agents.
  
  \vspace{0.2cm}
  
\textbf{Problem}: want $20\,\%$ children, $50\,\%$ adults/low-risk, $30\,\%$ elderly/high-risk
\newline
\ 
\end{onlyenv}
\begin{onlyenv}<3>
\begin{lstlisting}[basicstyle=\tiny\ttfamily, tabsize=6] 
N_AGENTS = 1000  
AGE_GROUPS = ["child", "adult", "elderly"] 
FRACTION_AGE_GROUPS = [0.2, 0.5, 0.3]  

class Agent:
	pass

def initialise():
	global agents
	agents = []

	for i in range(N_AGENTS):
		ag = Agent()
		ag.id = i			
		ag.health_state = "susceptible"	
		ag.age = np.random.choice(AGE_GROUPS, p=FRACTION_AGE_GROUPS) 
		agents.append(ag)
\end{lstlisting} 

  \vspace{0.2cm}
  
\textbf{Problem}: want $20\,\%$ children, $50\,\%$ adults/low-risk, $30\,\%$ elderly/high-risk
\newline
\ \Checkmark \ Draw from discrete probability distribution for each agent
\end{onlyenv}
\begin{onlyenv}<4>
\begin{lstlisting}[basicstyle=\tiny\ttfamily, tabsize=6] 
N_AGENTS = 1000
AGE_GROUPS = ["child", "adult", "elderly"] 
FRACTION_AGE_GROUPS = [0.2, 0.5, 0.3]  

class Agent:
	pass

def initialise():
	global agents
	agents = []

	for i in range(N_AGENTS):
		ag = Agent()
		ag.id = i			
		ag.health_state = "susceptible"	 	
		ag.age = np.random.choice(AGE_GROUPS, p=FRACTION_AGE_GROUPS) 
		agents.append(ag)
\end{lstlisting} 

  \vspace{0.2cm}

  \textbf{Problem}: want $20\,\%$ children, $50\,\%$ adults/low-risk, $30\,\%$ elderly/high-risk
\newline
\ \Checkmark \ Draw from discrete probability distribution for each agent \newline
\Put(130,280){\includegraphics[width=0.7\linewidth]{images/DistAge}}
\end{onlyenv}
\begin{onlyenv}<5>
\begin{lstlisting}[basicstyle=\tiny\ttfamily, tabsize=6] 

...
N_INFECTED_INIT = 2  # Number of agents in state "exposed" at t=0.

class Agent:
	pass

def initialise():
	global agents
	agents = []

	for i in range(N_AGENTS):
		ag = Agent()
		ag.id = i 
		ag.health_state = "susceptible"
		ag.age = np.random.choice(AGE_GROUPS, FRAC_AGE_GROUPS) 
		agents.append(ag)

	symptomatic_agents = np.random.choice(agents, size=N_INFECTED_INIT)
	for ag in symptomatic_agents:
		catch_virus(ag, t=0)

	return
\end{lstlisting} 
Infect a few randomly selected agents \newline (we will define function \textit{catch\_virus} in the next slides)
\end{onlyenv}
\end{overlayarea}
\end{frame}

\section{Concept 1: Drawing from discrete distributions}

%\subsection{Concept 1: Discrete Distributions}
\begin{frame}[fragile,t]
\frametitle{Concept 1: Discrete distributions -- a simple example}
\begin{example}
Problem: create a population with an attribute `sex' \newline \ \newline 
\visible<2->{\ \Checkmark \ For each agent, assign sex to male with $50\,\%$ and female else 
\newline}%
\visible<3->{\ \Checkmark \ For each agent, randomly draw one of two sexes (male/female with each $50\,\%$ probability) \newline \ra \textit{{\color{red} np.random.choice(options, size, replace, probability)}} \newline i.e.\ choose \textit{size} samples of \textit{options} with given \textit{probabilities}.}
\end{example}

\pause
\begin{lstlisting}[basicstyle=\tiny\ttfamily, tabsize=6]
for ag in agents:
    ag.sex = "male" if np.random.random() < 0.5 else "female"
\end{lstlisting}


\pause 
\begin{lstlisting}[basicstyle=\tiny\ttfamily, tabsize=6]
for ag in agents:
	ag.sex = np.random.choice(
		["male", "female"] # list/array of possible options
		p =[0.5, 0.5],     # probabilities assigned to these 
		size = 1,          # nr of draws (default=1)
            replace = True     # sample with/without replacement (default=True)
          )
\end{lstlisting}
%}

\end{frame}


\againframe<3->{Init}




\section{Concept 2: Drawing from continuous distributions}

\begin{frame}
  \frametitle{Concept 2: Continuous distributions}
	\begin{block}{Random variable}
	\begin{overlayarea}{\linewidth}{0.8\textheight}
	\only<1->{
	\begin{itemize}
	\item Random variable $x$
	\item Probability density function PDF: $p(x)$
	\item Needs to integrate to one: $\int_{-\infty}^{\infty} \ p(x) \ dx = 1$
	\item Now, we draw samples from this distribution
	\end{itemize}
	}
	%\includemovie{5cm}{2cm}{images/drawingfromdist.gif}
	%\animategraphics[loop, autoplay, width=0.9\textwidth]{1}{images/draw-}{0}{9} % TODO! THIS DOESN'T WORK.
	\only<2>{\includegraphics[width=0.9\textwidth]{images/draw-0}}
	\only<3>{\includegraphics[width=0.9\textwidth]{images/draw-1}}
	\only<4>{\includegraphics[width=0.9\textwidth]{images/draw-2}}
	\only<5>{\includegraphics[width=0.9\textwidth]{images/draw-3}}
	\end{overlayarea}
	\end{block}
\end{frame}

\begin{frame}
  \frametitle{Concept 2: Repetition}
	We have already applied `Concept 2: Drawing from continuous distributions' in both previous ABMs in Lectures 9 and 10: \\ \ \\
	Agents (e.g.\ foxes and rabbits) were randomly spawned on a 2D space $(x, y)$ with $x, y \, \in \, [0, 1]$.
        \newline
        Notation: \textit{Uniform distribution} between 0 and 1.
%	Each allowed value for the random variables $x$ and $y$ (i.e.\ between $0$ and $1$) is equally likely.
	\begin{center}
	\includegraphics[width=0.4\linewidth]{images/uniform_xy}
	\end{center}
	
\end{frame}



\begin{frame}[t]
  \frametitle{Concept 2: Common continuous distributions}
%	\begin{overlayarea}{\linewidth}{12cm}
\only<1>{
	\begin{block}{Uniform distribution}
	\includegraphics[width=0.9\textwidth]{images/UnifDist}
	
	\begin{tabularx}{\textwidth}{L{1.2cm}|X}
		What & continuous, bounded range (max and min are known) \\ \hline
		PDF & $\mathcal{U}(x_{\rm min}, x_{\rm max}) = \frac{1}{x_{\rm max} - x_{\rm min}}$  \\ \hline
		Usage & Uninformative. Use when we have no clue about the random variable.  	
	\end{tabularx}
	\end{block}
	}
\only<2>{
	\begin{block}{Gaussian or Normal distribution}
	\includegraphics[width=0.9\textwidth]{images/GaussDist}
	
	\begin{tabularx}{\textwidth}{L{1.2cm}|X}
What & continuous, infinite range  \\ \hline
PDF & $\mathcal{N}(\mu, \sigma) = \frac{1}{\sqrt{2 \pi \sigma^2}} \cdot \exp \left(-\frac{(x-\mu)^2}{2\sigma^2}\right)$   \\ \hline
          Usage & often observed in nature \ra law of large numbers.  \\
          & \ra very easy to use analytically. \\
	\end{tabularx}
\end{block}
}

\only<3>{
	\begin{block}{Delta distribution}
	\includegraphics[width=0.9\textwidth]{images/DeltaDist}

\begin{tabularx}{\textwidth}{L{1.2cm}|X}
	What & continuous, inifinte/bounded range  \\ \hline
PDF & $\delta(x-\tilde{x}) =: \delta_{\tilde{x}} = \begin{cases}
	\infty \text{ if } x=\tilde{x} \\ 0 \text{ else}
\end{cases}$  \quad $\int \delta(x-\tilde{x}) {\rm d}x := 1$ \\ \hline
Usage & We are absolutely certain about the parameter $x$, e.g.\ $g=9.81\, m/s^2$! Typically, we simply fix $x=\tilde{x}$ 
\end{tabularx}
\end{block}
}

\only<4>{
\begin{block}{Gamma distribution}
\begin{columns}
	\begin{column}{0.35\textwidth}
		\includegraphics[width=1\linewidth]{images/GammaDist} 
		{\scriptsize$\Gamma(5.9807,\ 0.948)$ }
	\end{column}
	\begin{column}{0.6\textwidth}
				\begin{itemize} \item $x$ semi-bounded $[0, \infty]$
				\item often used as distribution "close to Gaussian with long tail" (e.g.\ salary of people) \end{itemize}
	\end{column}
\end{columns}
\end{block}

\begin{block}{Beta distribution}
\begin{columns}
	\begin{column}{0.35\textwidth}
		\includegraphics[width=1\linewidth]{images/BetaDist} 	
				{\scriptsize	${\rm Beta}(1,3)$} 
	\end{column}
	\begin{column}{0.6\textwidth}
          \begin{itemize}
          \item $x$ bounded between $[0,1]$
          \item often used for parameters that represent uncertain probabilities
         \end{itemize}
	\end{column}
\end{columns}
\end{block}

	{\tiny \ra There are soo many distributions
	\url{https://en.wikipedia.org/wiki/List_of_probability_distributions}}
}
\end{frame}

\begin{frame}[fragile]
  \frametitle{Concept 2: Python Package `scipy.stats'}
  \begin{overlayarea}{\textwidth}{\textheight}
   \begin{onlyenv}<1>
\begin{block}{`scipy.stats'}
\begin{itemize} 
\item \lstinline|import scipy.stats as stats|
\item Create distribution e.g.\ via \lstinline|stats.norm(mu, sigma)|
\item 
\item \phantom{Whey}
\end{itemize}
\end{block}
\textbf{Create distribution:}
\begin{lstlisting}[basicstyle=\tiny\ttfamily, tabsize=6]
import scipy.stats as stats

mu = 1.2
sigma = 0.2
some_normal_dist = stats.norm(mu, sigma)
\end{lstlisting}
\begin{itemize} 
\item For other distributions, simply replace `norm' with e.g.\ `beta' and look up what parameters you need to specify!
\item {\scriptsize (as always in python, documentation is your friend \url{https://docs.scipy.org/doc/scipy/reference/stats.html} incl.\ examples and explanations of the parameters to specify, \ldots)}
\end{itemize}
\end{onlyenv}
\begin{onlyenv}<2,3>
\begin{block}{`scipy.stats'}
\begin{itemize} 
\item \lstinline|import scipy.stats as stats|
\item Create distribution e.g.\ via \lstinline|stats.norm(mu, sigma)|
\item PDF via \lstinline|.pdf(x)|
\item \phantom{Whey}
\end{itemize}
\end{block}

%\begin{block}
\textbf{PDF:}
\begin{lstlisting}[basicstyle=\tiny\ttfamily, tabsize=6]
x = np.linspace(0,2)
plt.plot(x, some_normal_dist.pdf(x))
plt.title(f'PDF of a normal distribution with mu={mu}, sigma={sigma}')
\end{lstlisting}
\only<3->{\Put(145,10){\includegraphics[width=0.6\linewidth]{images/gauss}}}
\end{onlyenv}
\begin{onlyenv}<4->
\begin{block}{`scipy.stats'}
\begin{itemize} 
\item \lstinline|import scipy.stats as stats|
\item Create distribution e.g.\ via \lstinline|stats.norm(mu, sigma)|
\item PDF via \lstinline|.pdf(x)|
\item Sampling via \lstinline|.rvs(samplesize)|
\end{itemize}
\end{block}

\textbf{Draw samples from the distribution:}
\begin{lstlisting}[basicstyle=\tiny\ttfamily, tabsize=6]
samples = some_normal_dist.rvs(100)  # Argument = Nr of samples 
plt.hist(samples)
plt.xlabel("x")
plt.ylabel("frequency")
plt.title("Histogram of samples")
plt.show()
\end{lstlisting}

\only<5->{\Put(145,30){\includegraphics[width=0.6\linewidth]{images/gauss_samples}}}

\end{onlyenv}
\end{overlayarea}
\end{frame}

%\begin{frame}[nonumber]
%Back to the Agents and Covid-19!
% \end{frame}

\againframe{codeoverview}

\begin{frame}
\frametitle{Back to the agent}
So far: \textbf{general properties} for an agent. \newline
Now: \textbf{infection-specific properties}.
\pause
{\small 
\begin{enumerate}[<+->]
\item Time of exposure
\item Symptomatic or asymptomatic?
\item How long is the incubation period? (when do symptoms start?)
\item From when to when is the agent infectious?
\item Will the agent die from the infection?
\item How infectious is the agent? (Is the agent a superspreader?)
\end{enumerate}
}
\visible<2->{\includegraphics[height=0.35\textheight]{images/courseInfection_Corona_noData}}%
\end{frame}


\begin{frame}[fragile]
\frametitle{Catch the virus -- Determine course of infection}
\begin{overlayarea}{\linewidth}{12cm}
\begin{onlyenv}<1>
\begin{lstlisting}[basicstyle=\tiny\ttfamily, tabsize=6] 



def catch_virus(ag, t_exposure):
    ag.health_state = "exposed"  
    ag.t_e = t_exposure
\end{lstlisting}
\vspace{0.5cm}
\fbox{\textbf{1}:}\newline the agent \textit{ag} catches the virus (\textit{catch\_virus}) at time \textit{t=t\_exposure}, i.e.\ after being infected with the virus by another agent. \\
First, the \textit{health\_state} of the agent changes to \textit{"exposed"} and the time of infection is saved in the internal variable \textit{t\_e}.
\end{onlyenv}

\begin{onlyenv}<2>
\begin{lstlisting}[basicstyle=\tiny\ttfamily, tabsize=6] 
P_SYMPTOMATIC = {"child": 0.1, "adult": 0.5, "elderly": 0.8} 


def catch_virus(ag, t_exposure):
    ...
    p_s = P_SYMPTOMATIC[ag.age]
    ag.symptomatic = True if np.random.random() < p_s else False
\end{lstlisting}
\vspace{0.5cm}
\fbox{\textbf{2}:}\newline determine whether the infection is
\begin{itemize}
\item symptomatic (\textit{ag.symptomatic = True}) with probabiltiy \textit{p\_s} or
\item asymptomatic (\textit{ag.symptomatic = False})
\end{itemize}
(\,\HandRight \ sample from two options with some probability). \\
The probability depends on age (vulnerability).
\end{onlyenv}

\begin{onlyenv}<3>
\begin{lstlisting}[basicstyle=\tiny\ttfamily, tabsize=6] 
INCUBATION_PERIOD_DIST = stats.gamma(5.807, 0.948)


def catch_virus(ag, t_exposure):
    ...
    incubation_period = INCUBATION_PERIOD.rvs()
\end{lstlisting}
\vspace{0.3cm}
\fbox{\textbf{3}:}\newline determine incubation period by drawing a sample from a gamma distribution inferred from data by \citet{lauer_incubation_2020} \newline (\,\HandRight\ draw from continuous distr.). 
\begin{center}
	\includegraphics[width=0.6\textwidth]{images/GammaDist_}
\end{center}
{\tiny Note: For asymptomatic cases the incubation has no meaning. It's just used as a characteristic value to determine the infectiousness period (next slide).}
\end{onlyenv}

\begin{onlyenv}<4>
\begin{lstlisting}[basicstyle=\tiny\ttfamily, tabsize=6] 



def catch_virus(ag, t_exposure):
    ...
    if ag.symptomatic:
	# Symptomatic Case
	ag.t_onset_symptoms = ag.t_e + incubation_period
    else:
	# Asymptomatic Case
	ag.t_onset_symptoms = np.nan
\end{lstlisting}
\vspace{0.5cm}
\fbox{\textbf{3}:}\newline symptoms start after the incubation period, if \textit{ag}'s infection is symptomatic. If \textit{ag} is asymptomatic, simply ignore.
\end{onlyenv}
\begin{onlyenv}<5>
\begin{lstlisting}[basicstyle=\tiny\ttfamily, tabsize=6] 
TIME_I_PRESYMPT = 2
TIME_I_POSTSYMPT = 4

def catch_virus(ag, t_exposure):
    ...
    ag.infectious_period = [
        ag.t_e + incubation_period - TIME_I_PRESYMPT,
        ag.t_e + incubation_period + TIME_I_POSTSYMPT
    ]
\end{lstlisting}
\vspace{0.5cm}
\fbox{\textbf{4}:}\newline the agent is only infectious two days before and four days after the onset of symptoms (or, for asymptomatic cases, the theoretical onset). 

\begin{center}
  %	\includegraphics[width=0.7\linewidth]{images/single_agents_sickness}
\includegraphics[height=0.2\textheight]{images/courseInfection_Corona_noData}%
\end{center}
\end{onlyenv}

\begin{onlyenv}<6>
\begin{lstlisting}[basicstyle=\tiny\ttfamily, tabsize=6] 
CFR = {"child": 0.00001, "adult": 0.005, "elderly": 0.05} 


def catch_virus(ag, t_exposure):
	...
	if ag.symptomatic:
		# Symptomatic Case, might die
		p_d = CFR[ag.age]
		ag.fatal_outcome = True if np.random.random() < p_d else False
	else:
		# Asymptomatic Case, can not die
		ag.fatal_outcome = False
\end{lstlisting}
  \fbox{\textbf{5}:} \newline
  if symptomatic, agent \textit{ag} might die (\textit{ag.fatal\_outcome = True}). Here, asymptomatic agents do not die.
\\
The probability of dying depends strongly on age. \\
\end{onlyenv}

\begin{onlyenv}<7>
\begin{lstlisting}[basicstyle=\tiny\ttfamily, tabsize=6]
BASE_I = stats.beta(1, 3)


def catch_virus(ag, t_exposure):
	...
	ag.base_infectiousness = BASE_I.rvs() 	# max probability to infect others
\end{lstlisting}
\fbox{\textbf{6}:}\newline each agent has a \textit{ag.base\_infectiousness} ($\in [0,1]$), a scale-factor determining how likely the agent will infect others when they interact. \\
\begin{center}
	\includegraphics[width=0.5\textwidth]{images/BetaDist_} \newline
	{\scriptsize beta distribution: a few agents are highly infectious, most are barely infectious.}
\end{center}
{\footnotesize Later: infectiousness depends on stage/type of infection (a- or pre-symptomatic people tend to be less infectious than symptomatic people) \ra defined later}
%\end{onlyenv}
\end{onlyenv}

\begin{onlyenv}<8>
Summary
\begin{lstlisting}[basicstyle=\tiny\ttfamily, tabsize=6]
def catch_virus(ag, t_exposure):
    ag.health_state = "exposed"  
    ag.t_e = t_exposure

    p_s = P_SYMPTOMATIC[ag.group]
    ag.symptomatic = True if np.random.random() < p_s else False

    incubation_period = INCUBATION_PERIOD.rvs()
    
    ag.infectious_period = [
        ag.t_e + incubation_period - TIME_I_PRESYMPT,
        ag.t_e + incubation_period + TIME_I_POSTSYMPT
    ]
    
    if ag.symptomatic:
        ag.t_onset_symptoms = ag.t_e + incubation_period
        p_d = CFR[ag.group]
        ag.fatal_outcome = True if np.random.random() < p_d else False
    else:
        ag.t_onset_symptoms = np.nan
        ag.fatal_outcome = False
        
    ag.base_infectiousness = BASE_I.rvs()   # * FACTOR_INFECTIOUSNESS
	
	return 
\end{lstlisting}
\end{onlyenv}
\end{overlayarea}
\end{frame}




\begin{frame}
\frametitle{A few examples of infection courses [Assignment]}
\begin{enumerate}
	\item Create an agent
	\item The agent catches the virus (i.e.\ call \textit{catch\_virus})
	\item Plot the course of the infection. 
	\item Repeat 1-3
\end{enumerate}

\pause
\begin{center}
	\includegraphics[width=0.9\linewidth]{images/agents_sickness}
\end{center}
\end{frame}


%%% Local Variables:
%%% mode: latex
%%% TeX-master: "CoronaABM"
%%% End:
