% !TeX root = CoronaABM.tex

%\begin{frame}
%	\frametitle{Observe Function}
%	Don't show code in detail, just results \ra next slides.\\
%	Show how to get these aggregate results (list comprehension and ask for state, then count states)
%\end{frame}

\begin{frame}[fragile]
	\frametitle{Run and Observe Function}

Goal: Perform simulation and track how many agents are in state "susceptible", "exposed", ... over time.
\begin{overlayarea}{\linewidth}{12cm}
\begin{onlyenv}<1>
\begin{lstlisting}[basicstyle=\tiny\ttfamily, tabsize=6] 
initialise()  # Initialise all agents
Net = initialise_network(agents, k=K_WS_NETWORK, p=P_WS_NETWORK)

T_ARRAY = np.linspace(0, 100, 0.5)

states = ["susceptible", "exposed", "inf_asympt", "inf_presympt", "inf_sympt", "recovered", "dead"]
results = np.empty([len(T_ARRAY), len(states)])

for n, t in enumerate(T_ARRAY):
	update(t)
	
	states_of_agents = [ag.state for ag in agents]
	N_s = states_of_agents.count("susceptible")
	N_e = states_of_agents.count("exposed")
	...
	results[n, :] = np.array([N_s, N_e, N_ia, ...])
\end{lstlisting}
\end{onlyenv}
\begin{onlyenv}<2>
\begin{figure}
	\centering
	\includegraphics[width=0.8\linewidth]{AggregateResults_N-1000-2_WS(6,2e-01)_seed2}
	\caption{An outbreak in a network with k=6, p = 0.2, 1000 agents of which 2 are exposed at t=0.}
	\label{fig:aggregateresultsn-1000-2ws62e-01seed1}
\end{figure}
\end{onlyenv}
\begin{onlyenv}<3>
	\begin{figure}
		\centering
		\includegraphics[width=0.8\linewidth]{AggregateResults_N-1000-2_WS(6,2e-01)_seed3}
		\caption{A \textit{different} outbreak in a network with k=6, p = 0.2, 1000 agents of which 2 are exposed at t=0.}
		\label{fig:aggregateresultsn-1000-2ws62e-01seed1}
	\end{figure}
\end{onlyenv}
\begin{onlyenv}<4>
\begin{figure}
	\centering
	\includegraphics[width=0.8\linewidth]{AggregateResults_N-1000-2_WS(6,2e-01)_seed1}
	\caption{This simulation is {\color{red}stochastic}! Even though the same model setup is used it might not lead to an outbreak (a qualitatively difference!). Model output depends on the actual course of microscopic events.}
\end{figure}
 %This simulation is {\color{red}stochastic}! Model output depends on the actual course of microscopic events.
\end{onlyenv}
\end{overlayarea}
\end{frame}

\againframe<6>{overview}

\begin{frame}
	\frametitle{Validation}
How can we verify if these results make sense? E.g.\ 
\bi 
\item Ensemble Runs: Do many different simulation runs.
\item Observe aggregate indicators
\bi 
\item The initial reproductive number$R_0$. 
\item Serial Interval (not done here)
\ei
\ei
\pause%\begin{onlyenv}<2>
\includegraphics[width=0.7\linewidth]{R0values_seed10}
%\end{onlyenv}

\end{frame}

\section{Policies}
\begin{frame}{Policies}
	The major idea of such a model was to test the impact of local policies. \\
	E.g.\ What happens 
	\bi 
	\item if the number of contacts is reduced (due to a socially distancing public)?
	\item or if the people keep contacts within tight clusters (households, neighbours)?
	\item or if the agents reduce their $base\_infectiousness$, by wearing a mask?	
	\ei 
	
\end{frame}

\begin{frame}{Parameters of the Model}
\begin{table}
	\scriptsize
\begin{tabularx}{\textwidth}{>{\tiny}l|X|p{2cm}}
	Parameter & Description & current value \\ \hline
	FRAC\_RISKGROUPS & Percentage of each risk group & [0.2, 0.5, 0.3]  \\
	BASE\_I & Distribution of base infectiousness of the agents & beta(1, 3) \\
	P\_SYMPTOMATIC & Probability to develop symptoms  & [0.1, 0.5, 0.8] \\
	INCUBATION\_PERIOD & Distribution of the incubation time & gamma(5.807, 0.948)\\
	CFR & Case fatality ratio for each age group & [0.0001, 0.005, 0.05] \\
	I\_SYMPTOMATIC & Strength of infectiousness for a symptomatic infection & 1 \\ 
	I\_ASYMPTOMATIC & -"- for an asymptomatic infection &  0.2 \\  
	I\_PRESYMPTOMATIC& -"- for a pre-symptomatic infection &  0.5 \\   
	TIME\_I\_PRESYMPT & infectious days before symptom onset & 2  \\
	TIME\_I\_POSTSYMPT & infectious days after symptom onset & 4  \\
	K\_WS\_NETWORK & Number of (nearest) neighbours in networks for nodes &  6 \\
	P\_WS\_NETWORK & Probability of each link to be rewired &  0.2  \\
	N\_AGENTS & Nr of agents & 1000 \\ %  # Number of agents in total
	N\_INFECTED\_INIT & Nr of agents set to "exposed" at t=0 & 2  \\
\end{tabularx}
\end{table}

\end{frame}

\begin{frame}{Policy: Decrease $k$ of Network to $k=4$}
\begin{figure}
	\only<1>{\includegraphics[width=\linewidth]{../../../CoronaABM/Simplified_Corona/Figures/AggregateResults_N-1000-2_WS(4,2e-01)_seed2}}
	\only<2>{
			\includegraphics[width=\linewidth]{../../../CoronaABM/Simplified_Corona/Figures/AggregateResults_N-1000-2_WS(4,2e-01)_seed2}
			\Put(-40,80){\includegraphics[width=0.7\linewidth]{../../../CoronaABM/Simplified_Corona/Figures/R0values_seed2}}}
	\end{figure}
\end{frame}

