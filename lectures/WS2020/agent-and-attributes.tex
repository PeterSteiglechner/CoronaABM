\begin{frame}<1-2>[label=Init, fragile]
\frametitle{Agent and Initialisation}
\begin{overlayarea}{\linewidth}{12cm}
\begin{onlyenv}<1>
\begin{lstlisting}[basicstyle=\tiny\ttfamily, tabsize=6] 
	
	
class Agent:
	pass
\end{lstlisting}
An empty class.
\end{onlyenv}
\begin{onlyenv}<2>
\begin{lstlisting}[basicstyle=\tiny\ttfamily, tabsize=6] 
N_AGENTS = 1000  # Number of agents in total

class Agent:
	pass

def initialise():
	global agents
	agents = []

	for id in range(N_AGENTS):
		ag = Agent()
		ag.id = id					# Unique ID
		ag.state = "susceptible"	# health state	
		ag.group = ???				# Age-/Rsik Group 
		agents.append(ag)
\end{lstlisting} 
Create all agents. \newline
\textbf{Problem}:   We want e.g. 20\% children, 50\% adults, 30\% high-risk people in our society.
\newline
\ 
\end{onlyenv}
\begin{onlyenv}<3>
\begin{lstlisting}[basicstyle=\tiny\ttfamily, tabsize=6] 
N_AGENTS = 1000  # Number of agents in total

#  Group: 	0 = children (no risk)  	--> 20%
#		1 = adults (small risk)		--> 50%
#		2 = high-risk group		--> 30%
RISKGROUPS = [0, 1, 2] 
FRAC_RISKGROUPS = [0.2, 0.5, 0.3]  

class Agent:
	pass

def initialise():
	global agents
	agents = []

	for i in range(N_AGENTS):
		ag = Agent()
		ag.id = i				# Unique ID
		ag.state = "susceptible"	# health state	
		ag.group = np.random.choice(RISKGROUPS, FRAC_RISKGROUPS) 
		agents.append(ag)
\end{lstlisting} 
\ \newline
\textbf{Problem}:  We want e.g. 20\% children, 50\% adults, 30\% high-risk people in our society.
\newline
\quad \Checkmark Draw from discrete probability distribution for each agent.
\end{onlyenv}
\begin{onlyenv}<4>
\begin{lstlisting}[basicstyle=\tiny\ttfamily, tabsize=6] 
N_AGENTS = 1000  # Number of agents in total

#  Group: 	0 = children (no risk)  	--> 20%
#		1 = adults (small risk)		--> 50%
#		2 = high-risk group		--> 30%
RISKGROUPS = [0, 1, 2] 
FRAC_RISKGROUPS = [0.2, 0.5, 0.3]  

class Agent:
	pass

def initialise():
	global agents
	agents = []

	for i in range(N_AGENTS):
		ag = Agent()
		ag.id = i				# Unique ID
		ag.state = "susceptible"	# health state	
		ag.group = np.random.choice(RISKGROUPS, FRAC_RISKGROUPS) 
		agents.append(ag)
\end{lstlisting} 
\ \newline
\textbf{Problem}:  We want e.g. 20\% children, 50\% adults, 30\% high-risk people in our society.
\newline
\quad \Checkmark Draw from discrete probability distribution for each agent. \newline
\Put(60,280){\includegraphics[width=0.9\linewidth]{images/DistGroups}}
\end{onlyenv}
\begin{onlyenv}<5>
\begin{lstlisting}[basicstyle=\tiny\ttfamily, tabsize=6] 
N_AGENTS = 1000  # Number of agents in total
N_INFECTED_INIT = 2  # Number of agents in state "exposed" at t=0.

class Agent:
	pass

def initialise():
	global agents
	agents = []

	for id in range(N_AGENTS):
		ag = Agent()
		ag.id = id					# Unique ID
		ag.state = "susceptible"	# health state	
		ag.group = np.random.choice(RISKGROUPS, FRAC_RISKGROUPS) 
		agents.append(ag)

	# Let N_INFECTED_INIT randomly chosen agents catch the virus at t=0
	symptomatic_agents = np.random.choice(agents, size=N_INFECTED_INIT)
	for ag in symptomatic_agents:
		catch_virus(ag, 0)		# agent ag catches virus at time 0.

	return
\end{lstlisting} 
Infect a few randomly selected agents. (we'll define function catch\_virus in the next slides)
\end{onlyenv}
\end{overlayarea}
\end{frame}


%\begin{frame}[fragile]
%\frametitle{The agent}
%\begin{overlayarea}{\linewidth}{12cm}
%\begin{onlyenv}<1>
%\begin{lstlisting}[basicstyle=\tiny\ttfamily, tabsize=6] 
%class Agent:
%	""" define an agent and its properties """
%\end{lstlisting}
%\end{onlyenv}
%\begin{onlyenv}<2>
%\begin{lstlisting}[basicstyle=\tiny\ttfamily, tabsize=6] 
%class Agent:
%	""" define an agent and its properties """
%	def __init__(self, _id):
%		""" initialise Agent """
%		self.id = _id		# Unique ID of an agent, Track the object!
%		self.state = "susceptible"	 	# health: susceptible to the virus
%\end{lstlisting} 
%	Any (more complex) class should have a "init"-function.\\
%	When we create an object of that class, the function "self.\_\_init\_\_()" is automatically called (with the arguments provided, here \_id=0). \\
%\begin{lstlisting}[basicstyle=\tiny\ttfamily, tabsize=6] 
%ag = Agent(0)	# implicitly self.__init__(0) is called
%ag.id == 0 
%\end{lstlisting} 
%The first argument, "self", refers to the "self" of the current object, does not need to be passed.\\
%To retrieve a variable (or call a function) \textit{within} the class, we need to refer to "self": 
%\begin{lstlisting}[basicstyle=\tiny\ttfamily, tabsize=6] 
%self.id
%self.__init__()
%\end{lstlisting} 	
%\end{onlyenv}
%\begin{onlyenv}<3>
%\begin{lstlisting}[basicstyle=\tiny\ttfamily, tabsize=6] 
%class Agent:
%	""" define an agent and its properties """
%	def __init__(self, _id):
%		""" initialise Agent """
%		self.id = _id		# Unique ID of an agent
%		self.state = "susceptible"	 	# health: susceptible to the virus
%		self.group =  		# age-/risk- group of the agent
%\end{lstlisting} 
%\textbf{Problem}: \newline
%\quad We want e.g. 20\% children, 50\% adults, 30\% elderly in our society. How can we assign a group to any random agent to obtain such a society?
%\newline
%\quad \Checkmark Draw from probability distribution whenever a new agent is initiated.
%\end{onlyenv}
%\end{overlayarea}
%\end{frame}

\section{Concept 1: Drawing from Discrete Distributions}
%\subsection{Concept 1: Discrete Distributions}
\begin{frame}[fragile]
\frametitle{Concept 1: Discrete Distributions: Simple Example}
\begin{example}
Problem: We want e.g. equal male and female proportions (50\% each) in our society. How can we assign a sex to any random agent to obtain such a society? \newline  \ \newline 
\pause \quad \Checkmark For each agent randomly draw a sex with probability 50\% \ra Use the function \textit{\color{red} np.random.choice(options, probability, size)} that chooses between options with given probabilities.
\end{example}

\begin{lstlisting}[basicstyle=\tiny\ttfamily, tabsize=6]
for ag in agents:
	ag.sex = np.random.choice(
		["male", "female"]	# List/Array of possible options
		p=[0.5, 0.5], 		# Probabilities assigned to these 
		size=1			# Number of draws (default=1)
	)
\end{lstlisting}
\end{frame}

\againframe<3->{Init}

%\begin{frame}[fragile]
%\frametitle{The agent - groups from distribution}
%\begin{overlayarea}{\linewidth}{11cm}
%\begin{onlyenv}<1->
%\begin{lstlisting}[basicstyle=\tiny\ttfamily, tabsize=6] 
%#  Group: 	0 = children (risk-free)  	--> 20%
%#		1 = adults (small risk)		--> 50%
%#		2 = high-risk group		--> 30%
%RISKGROUPS = [0, 1, 2] 
%FRAC_RISKGROUPS = [0.2, 0.5, 0.3]  
%
%class Agent:
%	""" define an agent and its properties """
%	def __init__(self, _id):
%		""" initialise Agent """
%		self.id = _id	# Unique ID of an agent
%		self.state = "susceptible"	 	# health: susceptible to the virus
%		self.group = np.random.choice(RISKGROUPS, p=FRAC_RISKGROUPS)				
%\end{lstlisting} 
%\end{onlyenv}
%\begin{onlyenv}<2->
%	\centering
%	\includegraphics[width=0.9\linewidth]{images/DistGroups}
%\end{onlyenv}
%\end{overlayarea}
%\end{frame}



\section{Concept 2: Drawing from Continuous Distributions}

\begin{frame}{Concept 2: Continuous Distributions}
	\begin{block}{Random Variable}
	\begin{overlayarea}{\linewidth}{12cm}
	\only<1->{
	\bi
	\item Random Variable $x$
	\item Probability Distribution (or Probability Density Function PDF): $p(x)$ or ${\rm dist}_{\rm PDF}(x)$
	\item Needs to integrate to 1: $\int_{-\infty}^{\infty} \ p(x) \ dx = 1$
	\item Now we simply sample from this distribution!
	\ei
	}
	%\includemovie{5cm}{2cm}{images/drawingfromdist.gif}
	%\animategraphics[loop, autoplay, width=0.9\textwidth]{1}{images/draw-}{0}{9} % TODO! THIS DOESN'T WORK.
	\only<2>{\includegraphics[width=0.9\textwidth]{images/draw-0}}
	\only<3>{\includegraphics[width=0.9\textwidth]{images/draw-1}}
	\only<4>{\includegraphics[width=0.9\textwidth]{images/draw-2}}
	\only<5>{\includegraphics[width=0.9\textwidth]{images/draw-3}}
	\end{overlayarea}
	\end{block}
\end{frame}

\begin{frame}{Concept 2: Repetition}
	This is exactly what we have done for both previous ABMs in Lecture 10 and 11: \\ \ \\
	Agents (e.g.\ foxes and rabbits) were randomly spawned on a 2D space $(x, y)$ with $x, \, y \ \in \  [0, 1]$.
	This is called the \textit{Uniform Distribution}. Each allowed value for the random variables $x$ and $y$ (i.e.\ between $0$ and $1$) is equally likely.
	\begin{center}
	\includegraphics[width=0.3\linewidth]{images/uniform_xy}
	\end{center}
	
\end{frame}



\begin{frame}{Concept 2: Common cont.\ distributions}
%	\begin{overlayarea}{\linewidth}{12cm}
\only<1>{
	\begin{block}{Uniform Distribution}
	\includegraphics[width=0.9\textwidth]{images/UnifDist}
	
	\begin{tabularx}{\textwidth}{L{1.3cm}|L{6.5cm}|L{2cm}}
		What & continuous, bounded range (maximum and minimum is known) & \\ \hline
		PDF & $\mathcal{U}(x_{\rm min}, x_{\rm max}) = \frac{1}{x_{\rm max} - x_{\rm min}}$ & \\ \hline
		Usage & Uninformative, when nothing is known about the random variable & 	
	\end{tabularx}
	\end{block}
	}
\only<2>{
	\begin{block}{Gaussian/Normal Distribution}
	\includegraphics[width=0.9\textwidth]{images/GaussDist}
	
	\begin{tabularx}{\textwidth}{L{1.3cm}|L{6.5cm}|L{2cm}}
What & continuous, infinite range & \\ \hline
PDF & $\mathcal{N}(\mu, \sigma) = \frac{1}{\sqrt{2 \pi \sigma^2}} \cdot \exp \left(-\frac{(x-\mu)^2}{2\sigma^2}\right)$ & \ra very easy to use analytically \\ \hline
Usage & Most often observed in nature \ra Law of large numbers. & \\	
	\end{tabularx}
\end{block}
}

\only<3>{
	\begin{block}{Delta Distribution}
	\includegraphics[width=0.9\textwidth]{images/DeltaDist}

\begin{tabularx}{\textwidth}{L{1.3cm}|L{6.5cm}|L{2cm}}
	What & continuous, inifinte/bounded range & \\ \hline
PDF & $\delta(\tilde{x}) = \delta_{\tilde{x}} = \begin{cases}
	\infty \text{ if } x=\tilde{x} \\ 0 \text{ else}
\end{cases}$ & (integral is 1 per def.)\\ \hline
Usage & We know the value for sure! We use this whenever we set a parameter to a fixed value without uncertainty. & 
\end{tabularx}
\end{block}
}

\only<4>{
\begin{block}{Gamma Distribution}
\begin{columns}
	\begin{column}{0.3\textwidth}
		\includegraphics[width=1\textwidth]{images/GammaDist} 
		{\scriptsize$\Gamma(5.9807,\ 0.948)$ }
	\end{column}
	\begin{column}{0.7\textwidth}
				\bi \item $x$ semi-bounded $[0, \infty]$
				\item often used as distribution "close to Gaussian with long tail" (e.g.\ salary of people) \ei
	\end{column}
\end{columns}
\end{block}

\begin{block}{Beta Distribution}
\begin{columns}
	\begin{column}{0.3\textwidth}
		\includegraphics[width=1\textwidth]{images/BetaDist} 	
				{\scriptsize	${\rm Beta}(1,3)$} 
	\end{column}
	\begin{column}{0.6\textwidth}
	\bi \item $x$ bounded between $[0,1]$ \ei
	\end{column}
\end{columns}
\end{block}

	{\tiny \ra There're soo many distributions. Wikipedia is informative and well structured 
	\url{https://en.wikipedia.org/wiki/List_of_probability_distributions}}
}
\end{frame}

\begin{frame}[fragile]
\frametitle{Concept 2: Python Package "scipy.stats"}
\begin{onlyenv}<1>
\begin{block}{Scipy Stats}
\bi 
\item \lstinline|import scipy.stats as stats|
\item Create distribution e.g. via \lstinline|stats.norm(mu, sigma)|
\item 
\item 
\ei
\end{block}
\textbf{Create distribution:}
\begin{lstlisting}[basicstyle=\tiny\ttfamily, tabsize=6]
import scipy.stats as stats

mu = 1.2
sigma = 0.2
some_normal_dist = stats.norm(mu, sigma)
\end{lstlisting}
\bi 
\item For other distributions simply replace "norm" e.g. with "beta" and look up what parameters you need to specify!! 
\item {\scriptsize (as always in python) \url{https://docs.scipy.org/doc/scipy/reference/stats.html} 
(incl examples and explanations of the parameters to specify, \ldots)}
\ei
\end{onlyenv}
\begin{onlyenv}<2>
\begin{block}{Scipy Stats}
\bi 
\item \lstinline|import scipy.stats as stats|
\item Create distribution e.g. via \lstinline|stats.norm(mu, sigma)|
\item Probability Density Function via \lstinline|.pdf(x)|
\item 
\ei
\end{block}

%\begin{block}
\textbf{Get the PDF:}
\begin{lstlisting}[basicstyle=\tiny\ttfamily, tabsize=6]
x = np.linspace(-2,2)
plt.plot(x, some_normal_dist.pdf(x))
plt.title('PDF of a normal distribution with mu='+str(mu)+', sigma='+str(sigma))
plt.show()
\end{lstlisting}
\begin{center}
	\includegraphics[width=0.6\linewidth]{images/GaussDist_2}
\end{center}
\end{onlyenv}
\begin{onlyenv}<3>
\begin{block}{Scipy Stats}
\bi 
\item \lstinline|import scipy.stats as stats|
\item Create distribution e.g. via \lstinline|stats.norm(mu, sigma)|
\item PDF via \lstinline|.pdf(x)|
\item Sampling via \lstinline|.rvs(samplesize)|
\ei
\end{block}

\textbf{Draw samples from the distribution:}
\begin{lstlisting}[basicstyle=\tiny\ttfamily, tabsize=6]
samples = some_normal_dist.rvs(100)  # Argument = Nr of samples 
plt.hist(samples)
plt.xlabel("x")
plt.ylabel("Frequency")
plt.title("Histogram of samples")
plt.show()
\end{lstlisting}

\begin{center}
		\includegraphics[width=0.6\linewidth]{images/rvs}
\end{center}

\end{onlyenv}
\end{frame}

%\begin{frame}[nonumber]
%Back to the Agents and Covid-19!
%\end{frame}

\begin{frame}[fragile]
\frametitle{Back to the agent}
\begin{overlayarea}{\linewidth}{12cm}
\begin{onlyenv}<1-7>
%\begin{lstlisting}[basicstyle=\tiny\ttfamily, tabsize=6] 
%class Agent(_id):
%	""" define an agent and its properties """
%	def __init__(self, _id):
%		""" initialise Agent """
%
%		# General Properties
%		self.id = _id	# Unique ID of an agent
%		self.state = "susceptible"	 	# health: susceptible to the virus
%		self.group = np.random.choice(RISKGROUPS, p=FRAC_RISKGROUPS)	
%\end{lstlisting}
We have defined some \textbf{general properties} for an agent. When the agent is infected, it will need further, \textbf{infection-specific properties}:
\be
\item<2-7> Time of exposure
\item<3-7> Symptomatic or asymptomatic?
\item<4-7> How long is the incubation period? When do symptoms begin?
\item<5-7> From when to when is the agent infectious?
\item<6-7> Will the agent die from the infection?
\item<7-7> How infectious is the agent? Is the agent a superspreader?
\ee
\end{onlyenv}

%\begin{onlyenv}<8>
%\begin{lstlisting}[basicstyle=\tiny\ttfamily, tabsize=6] 
%class Agent(_id):
%	""" define an agent and its properties """
%	def __init__(self, _id):
%		""" initialise Agent """
%		
%		# General Properties
%		self.id = _id	# Unique ID of an agent
%		self.state = "susceptible"	 	# health: susceptible to the virus
%		self.group = np.random.choice(RISKGROUPS, p=FRAC_RISKGROUPS)	
%		
%		# infection specific attributes (for later)
%		# np.nan --> "not a number", i.e. they are not yet defined.
%		
%		# (float) time of catching the virus
%		self.t_e = np.nan  			
%		# (Bool) whether infection is symptomatic or not
%		self.symptomatic = np.nan	
%		# (float) time of onset of symptoms
%		self.t_onset_symptoms = np.nan	
%		# (Bool) whether the agent dies from the infection
%		self.fatal_case = np.nan	
%		# ([float, float]) start and end time of the infectious period
%		self.infectious_period = [np.nan, np.nan]
%		# (float in [0,1]) how infectious is the agent	
%		self.base_infectiousness = np.nan
%		
%		return 
%\end{lstlisting}
%\end{onlyenv}
\end{overlayarea}
\end{frame}


\begin{frame}[fragile]
\frametitle{Catch the virus -- Determine course of infection}
\begin{overlayarea}{\linewidth}{12cm}
\begin{onlyenv}<1>
\begin{lstlisting}[basicstyle=\tiny\ttfamily, tabsize=6] 



def catch_virus(ag, t_exposure):
    ag.state = "exposed"  
    ag.t_e = t_exposure
\end{lstlisting}
\vspace{0.5cm}
\fbox{\textbf{1}:} The agent \textit{ag} catches the virus (\textit{catch\_virus})at time \textit{t=t\_exposure}, i.e.\ after being infected with the virus by another agent.
First, the \textit{state} of the agent changes to "exposed" and the time of infection is saved in the internal variable \textit{t\_e}.
\end{onlyenv}

\begin{onlyenv}<2>
\begin{lstlisting}[basicstyle=\tiny\ttfamily, tabsize=6] 
P_SYMPTOMATIC = [0.1, 0.5, 0.8] 


def catch_virus(ag, t_exposure):
	...
    p_s = P_SYMPTOMATIC[ag.group]
    ag.symptomatic = np.random.choice([True, False], p=[p_s, 1 - p_s])
\end{lstlisting}
\vspace{0.5cm}
\fbox{\textbf{2}:} We determine randomly whether the course of the infection is going to be symptomatic (\textit{ag.symptomatic = True}) or asymptomatic (\textit{ag.symptomatic = False}) (\HandRight\ \  Discrete Distribution for two options). \\
The probability depends on the agent's age-/riskgroup.
\end{onlyenv}
\begin{onlyenv}<3>
\begin{lstlisting}[basicstyle=\tiny\ttfamily, tabsize=6] 
INCUBATION_PERIOD_DIST = stats.gamma(5.807, 0.948)


def catch_virus(ag, t_exposure):
	...
	incubation_period = INCUBATION_PERIOD.rvs()
\end{lstlisting}
\vspace{0.5cm}
\fbox{\textbf{3}:} Draw one incubation period from gamma distribution (fitted to data by \citet{lauer_incubation_2020}) (\HandRight\ \  Draw from continuous  distr.). 
\begin{center}
	\includegraphics[width=0.6\textwidth]{images/GammaDist_}
\end{center}
{\tiny Note: For asymptomatic cases the incubation has no meaning. It's just used as a characteristic value to for the infectiousness period (next slide).}
\end{onlyenv}

\begin{onlyenv}<4>
\begin{lstlisting}[basicstyle=\tiny\ttfamily, tabsize=6] 



def catch_virus(ag, t_exposure):
	...
	if ag.symptomatic:
		# Symptomatic Case
		ag.t_onset_symptoms = ag.t_e + incubation_period
	else:
		# Asymptomatic Case
		ag.t_onset_symptoms = np.nan
\end{lstlisting}
\vspace{0.5cm}
\fbox{\textbf{3}:} Symptom onset after incubation period (if \textit{ag}'s infection will be symptomatic), otherwise, ignore.
\end{onlyenv}
\begin{onlyenv}<5>
\begin{lstlisting}[basicstyle=\tiny\ttfamily, tabsize=6] 
TIME_I_PRESYMPT = 2
TIME_I_POSTSYMPT = 4

def catch_virus(ag, t_exposure):
	...
	ag.infectious_period = [
        ag.t_e + incubation_period - TIME_I_PRESYMPT,
        ag.t_e + incubation_period + TIME_I_POSTSYMPT
    ]
\end{lstlisting}
\vspace{0.5cm}
\fbox{\textbf{4}:} The agent is only infectious a few days before and after the onset of symptoms. \\
Here, I assume, that the period in which \textit{asymptomatic} people are infectiousness is similarly distributed as for the symptomatic cases. \\
\begin{center}
	\includegraphics[width=0.7\linewidth]{images/single_agents_sickness}
\end{center}
\end{onlyenv}

\begin{onlyenv}<6>
\begin{lstlisting}[basicstyle=\tiny\ttfamily, tabsize=6] 
CFR = [0.0001, 0.005, 0.05] 	# for each age-/riskgroup


def catch_virus(ag, t_exposure):
	...
	if ag.symptomatic:
		# Symptomatic Case, might die
		p_d = CFR[ag.group]
		ag.fatal_case = np.random.choice([True, False], p=[p_d, 1-p_d])
	else:
		# Asymptomatic Case, can not die
		ag.fatal_case = False
\end{lstlisting}
\fbox{\textbf{5}:} If the agent is symptomatic, there's a chance the agent might die (\textit{ag.fatal\_case = True}).\\
The probability depends strongly on the age-/riskgroup. \\
Asymptomatic agents do not die in this model.
\end{onlyenv}

\begin{onlyenv}<7>
\begin{lstlisting}[basicstyle=\tiny\ttfamily, tabsize=6]
BASE_I = stats.beta(1, 3)


def catch_virus(ag, t_exposure):
	...
	ag.base_infectiousness = BASE_I.rvs() 	# factor in probability to infect others
\end{lstlisting}
\fbox{\textbf{6}:} Each agent has a heterogeneous \textit{ag.base\_infectiousness} ($\in [0,1]$), which factors in the probability of infecting another agents when they are in contact and, thus, exposing them. \\
\begin{center}
	\includegraphics[width=0.7\textwidth]{images/BetaDist_} \newline
	{\scriptsize Drawn from a beta distribution: few agents are highly, but most hardly infectious.}
\end{center}
The actual probability depends also on the time or type of infection (typically a- or pre-symptomatic < symptomatic infectiousness(?)) \ra defined later.
%\end{onlyenv}
\end{onlyenv}

\begin{onlyenv}<8>
Summary
\begin{lstlisting}[basicstyle=\tiny\ttfamily, tabsize=6]
def catch_virus(ag, t_exposure):
    ag.state = "exposed"  
    ag.t_e = t_exposure

    p_s = P_SYMPTOMATIC[ag.group]
    ag.symptomatic = np.random.choice([True, False], p=[p_s, 1 - p_s])

    incubation_period = INCUBATION_PERIOD.rvs()
    
    ag.infectious_period = [
        ag.t_e + incubation_period - TIME_I_PRESYMPT,
        ag.t_e + incubation_period + TIME_I_POSTSYMPT
    ]
    
    if ag.symptomatic:
        ag.t_onset_symptoms = ag.t_e + incubation_period
        p_d = CFR[ag.group]
        ag.fatal_case = np.random.choice([True, False], p=[p_d, 1 - p_d])
    else:
        ag.t_onset_symptoms = np.nan
        ag.fatal_case = False
        
    ag.base_infectiousness = BASE_I.rvs()   # * FACTOR_INFECTIOUSNESS
	
	return 
\end{lstlisting}
\end{onlyenv}
\end{overlayarea}
\end{frame}


%\begin{frame}[fragile]
%\frametitle{The agent -- Determine course of infection}
%\begin{overlayarea}{\linewidth}{12cm}
%\begin{onlyenv}<1>
%\begin{lstlisting}[basicstyle=\tiny\ttfamily, tabsize=6] 
%	
%	
%	
%class Agent(_id):
%	# ... 
%	def __init__(self, _id):
%		# ...
%		
%	def catch_virus(self, t_exposure):	
%		self.state = "exposed"  # Exposed
%		self.t_e = t_exposure
%\end{lstlisting}
%\fbox{\textbf{1}:} An agent triggers the internal function "catch\_virus" at time t=t\_exposure, i.e.\ after being infected with the virus.
%First the state of the agent changes to "exposed" and the time of infection is saved in the internal variable "t\_e"
%\end{onlyenv}
%\begin{onlyenv}<2>
%\begin{lstlisting}[basicstyle=\tiny\ttfamily, tabsize=6]
%P_SYMPTOMATIC = [0.1, 0.5, 0.8] 
%
%
%class Agent(_id):
%	# ... 
%	def catch_virus(self, t_exposure):	
%		self.state = "exposed"  	# Exposed
%		self.t_e = t_exposure
%		
%		# Probability of infection being symptomatic (dependent on agegroup)
%		p_s = P_SYMPTOMATIC[self.group]
%		self.symptomatic = np.random.choice(  
%							[True, False],
%							p=[p_s, 1 - p_s]
%						)
%\end{lstlisting}
%\fbox{\textbf{2}:} We determine randomly whether the course of the infection is going to be symptomatic (self.symptomatic = True) or asymptomatic (self.symptomatic = False) (\HandRight\ \  Discrete Distribution for two options). \\
%The probability depends on the agent's age-/riskgroup.
%\end{onlyenv}
%\begin{onlyenv}<3>
%\begin{lstlisting}[basicstyle=\tiny\ttfamily, tabsize=6]
%INCUBATION_PERIOD_DIST = stats.gamma(5.807, 0.948)
%	
%	
%class Agent(_id):
%	# ... 
%	def catch_virus(self, t_exposure):
%		# ...
%		incubation_period = INCUBATION_TIME_DIST.rvs()
%\end{lstlisting}
%\fbox{\textbf{3}:} We draw one incubation period from the continuous gamma distribution (that was fitted to data by \citet{lauer_incubation_2020}). 
%\fbox{\textbf{4}:} We draw one incubation period from the continuous gamma distribution (that was fitted to data by \citet{lauer_incubation_2020}). 

%\begin{center}
%	\includegraphics[width=0.7\textwidth]{images/GammaDist_}
%\end{center}
%\vfill
%{\tiny Note: For asymptomatic cases the incubation has no meaning. It's just used as a characteristic value to for the infectiousness period (next slide).}
%\begin{center}
%	\includegraphics[width=0.7\textwidth]{images/GammaDist_}
%\end{center}
%\vfill
%{\tiny Note: For asymptomatic cases the incubation has no meaning. It's just used as a characteristic value to for the infectiousness period (next slide).}
%\end{onlyenv}
%
%
%\begin{onlyenv}<4>
%\begin{lstlisting}[basicstyle=\tiny\ttfamily, tabsize=6]
%	
%	
%	
%class Agent(_id):
%	# ... 
%	def catch_virus(self, t_exposure):
%		# ...
%		if self.symptomatic:
%			# Symptomatic Case
%			self.t_onset_symptoms = self.t_e + incubation_period
%		else:
%			# Asymptomatic Case
%			self.t_onset_symptoms = np.nan
%\end{lstlisting}
%\fbox{\textbf{3}:} Symptom onset.
%\end{onlyenv}
%
%\begin{onlyenv}<5>
%\begin{lstlisting}[basicstyle=\tiny\ttfamily, tabsize=6]
%TIME_I_PRESYMPT = 2
%TIME_I_POSTSYMPT = 4
%
%class Agent(_id):
%	# ... 
%	def catch_virus(self, t_exposure):
%		# ...
%		self.infectious_period = [
%			self.t_e + incubation_period - TIME_I_PRESYMPT,
%			self.t_e + incubation_period + TIME_I_POSTSYMPT
%		]
%\end{lstlisting}
%\fbox{\textbf{4}:} The agent is only infectious a few days before and after the onset of symptoms. \\
%Here, I assume, that the period in which \textit{asymptotic} people are infectiousness is similarily distributed as for the symptomatic cases. \\
%\begin{center}
%	\includegraphics[width=0.7\linewidth]{images/single_agents_sickness}
%\end{center}
%
%\end{onlyenv}
%
%
%\begin{onlyenv}<6>
%\begin{lstlisting}[basicstyle=\tiny\ttfamily, tabsize=6]
%CFR = [0.0001, 0.005, 0.05] 	# for each age-/riskgroup
%
%
%class Agent(_id):
%	# ... 
%	def catch_virus(self, t_exposure):
%		# ...
%		if self.symptomatic:
%			# Symptomatic Case, might die
%			p_c = CFR[self.group]
%			self.fatal_case = np.random.choice([True, False], p=[p_c, 1-p_c])
%		else:
%			# Asymptomatic Case, can not die
%			self.fatal_case = False
%\end{lstlisting}
%\fbox{\textbf{5}:} If the agent is symptomatic, there's a chance she might die (self.fatal\_case = True).\\
%The probability depends strongly on the age-/riskgroup. \\
%Asymptomatic agents do not die in this model.
%\end{onlyenv}
%
%
%\begin{onlyenv}<7-8>
%\begin{lstlisting}[basicstyle=\tiny\ttfamily, tabsize=6]
%BASE_I = stats.beta(1, 3)
%
%
%class Agent(_id):
%	# ... 
%	def catch_virus(self, t_exposure):
%		# ...
%		self.base_infectiousness = min(1, BASE_I.rvs()
%\end{lstlisting}
%\fbox{\textbf{6}:} Each Agent has a specific, heterogeneous infectiousness ($\in [0,1]$), which determines how likely she will infect other agents when they are in contact. \\
%\end{onlyenv}
%\begin{onlyenv}<7>
%This base\_infectiousness is drawn from a beta distribution, such that a few agents are highly infectious, and most are barely infectious. \\
%
%The actual infectiousness depends on the time as well (typically pre-symptomatic < symptomatic infectiousness(?)) \ra defined later.
%\end{onlyenv}
%\begin{onlyenv}<8>
%\begin{center}
%	\includegraphics[width=0.7\textwidth]{images/BetaDist_}
%\end{center}
%\end{onlyenv}
%
%\begin{onlyenv}<9>
%Summary:
%\begin{lstlisting}[basicstyle=\tiny\ttfamily, tabsize=6]
%class Agent:
%	# ...
%	def catch_virus(self, t_exposure):
%		self.state = "exposed"  #
%		self.t_e = t_exposure
%		
%		p_s = P_SYMPTOMATIC[self.group]
%		self.symptomatic = np.random.choice([True, False], p=[p_s, 1 - p_s])
%			
%		incubation_period = INCUBATION_PERIOD.rvs()
%		self.infectious_period = [
%			self.t_e + incubation_period - TIME_I_PRESYMPT,
%			self.t_e + incubation_period + TIME_I_POSTSYMPT
%		]
%		
%		if self.symptomatic:
%			self.t_onset_symptoms = self.t_e + incubation_period
%			
%			p_c = CFR[self.group]
%			self.fatal_case = np.random.choice([True, False], p=[p_c, 1-p_c])
%		else:
%			self.t_onset_symptoms = np.nan
%			self.fatal_case = False
%		
%		self.base_infectiousness = min(1, BASE_I.rvs())  
%		return
%\end{lstlisting}
%\end{onlyenv}
%\end{overlayarea}
%\end{frame}




\begin{frame}
\frametitle{A few examples of infection courses}

\begin{enumerate}
	\item Create an agent
	\item Let it catch the virus (i.e.\ call \textit{catch\_virus})
	\item Plot the course of the infection. 
	\item Repeat action 1-3
\end{enumerate}

\pause
\begin{center}
	\includegraphics[width=0.9\linewidth]{images/agents_sickness}
\end{center}
\end{frame}


%\begin{frame}[fragile]
%	\frametitle{Initialise Agents}
%	\begin{overlayarea}{\linewidth}{12cm}
%		\begin{onlyenv}<1>
%			\begin{lstlisting}[basicstyle=\tiny\ttfamily, tabsize=6] 
%				N_AGENTS = 1000  # Number of agents in total
%				
%				
%				def initialise():
%				global agents
%				agents = []
%				
%				for _id in range(N_AGENTS):
%				ag = Agent(_id)
%				agents.append(ag)
%			\end{lstlisting}
%		Create Population of all agents
%		\end{onlyenv}
%		
%		\begin{onlyenv}<2>
%			\begin{lstlisting}[basicstyle=\tiny\ttfamily, tabsize=6] 
%				N_AGENTS = 1000  # Number of agents in total
%				N_INFECTED_INIT = 2  # Number of agents in state "exposed" at t=0.
%				
%				def initialise():
%				global agents
%				agents = []
%				
%				for _id in range(N_AGENTS):
%				ag = Agent(_id)
%				agents.append(ag)
%				
%				# Let N_INFECTED_INIT randomly chosen agents catch the virus at t=0
%				symptomatic_agents = np.random.choice(agents, size=N_INFECTED_INIT)
%				for ag in symptomatic_agents:
%				ag.catch_virus(0)
%				return
%			\end{lstlisting}
%		Initialise a few agents as exposed with the virus.
%		\end{onlyenv}
%	\end{overlayarea}
%\end{frame}
