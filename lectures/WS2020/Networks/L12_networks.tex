\documentclass[x11names,svgnames,table]{beamer}
\usetheme{sharky}

\usepackage[compatibility=false, justification=justified]{caption}    % to allow the use of \caption* i.e. caption without numbering

\usepackage{amssymb}
\usepackage[T1]{fontenc}

%\usepackage[scaled]{beramono} % allows the use of textbf under texttt

\usepackage{bbding} % for symbols such as \HandCuffRight

\usepackage{eqnarray,amsmath}
\usepackage{mathtools}

\usepackage{forloop}
\newcounter{ct} 
%\usepackage[table]{xcolor}  % to color rows of tables using \rowcolor{color} before the table row. 
                                                  % colors include red, blue, black, gray, yellow and many other popular colors.

\usepackage{ragged2e} % to justify itemized text

\usepackage{media9}   % this to embed youtube video in the pdf as flash file 

\usepackage{animate}  % to created animated png


% defining new commands
\newcommand\bhline{\arrayrulecolor{blue}\hline\arrayrulecolor{black}}
\newcommand\tg{\textcolor[gray]{0.8}}
\newcommand\tgr{\textcolor{green}}
\newcommand\tog{\textcolor{OliveGreen}}
\newcommand\tpg{\textcolor{PineGreen}}
\newcommand\tgo{\textcolor{Goldenrod}}
\newcommand\tro{\textcolor{RedOrange}}
\newcommand\tyo{\textcolor{YellowOrange}}
\newcommand\tbo{\textcolor{BurntOrange}}
\newcommand\ty{\textcolor{yellow}}
\newcommand\tr{\textcolor{red}}
\newcommand\tb{\textcolor{blue}}

\newcommand\bi{\begin{itemize}}
\newcommand\ei{\end{itemize}}
\newcommand\be{\begin{enumerate}}
\newcommand\ee{\end{enumerate}}
\newcommand\bce{\begin{center}}
\newcommand\ece{\end{center}}
\newcommand\beq{\begin{equation}}
\newcommand\eeq{\end{equation}}
\newcommand\bea{\begin{eqnarray}}
\newcommand\eea{\end{eqnarray}}

\newcommand\bfig{\begin{figure}}
\newcommand\efig{\end{figure}}

\usepackage{color}
\definecolor{mygreen}{rgb}{0,0.6,0}
\definecolor{mygray}{rgb}{0.5,0.5,0.5}
\definecolor{mymauve}{rgb}{0.58,0,0.82}
\definecolor{myblue}{rgb}{0.0, 0.18, 0.65}

\usepackage{listings} % for displaying programming code in LaTex (see: https://www.sharelatex.com/learn/Code_listing) 

    \lstset{basicstyle=\tiny\ttfamily,  % size and style of fonts used for the code
      backgroundcolor=\color{white},  % choose the background color
      breaklines=true,                          % break lines when too long 
      resetmargins=true,                      % reset the margins
      xleftmargin=-0.45cm,                 % set the left margin
      xrightmargin=0.0cm,                   % set the right margin
      showspaces=false,                       % show spaces adding particular underscores
      showstringspaces=false,             % underline spaces within strings
      showtabs=false,                          % show tabs within strings adding 
      escapechar=@,                             % to add LaTeX commands within the code
      keywordstyle=\color{mymauve},  % keyword style
      commentstyle=\color{red},           % comment style
      %stringstyle=\color{mygreen}       % string literal style
} 

\usepackage{hyperref}
\hypersetup{
    colorlinks=true,
    linkcolor=blue,
    filecolor=blue,      
    urlcolor=blue,
}
\urlstyle{same}

\usepackage{changepage}
\usepackage{bold-extra}


\title[{\color{white}Lecture 11~~--~~Network Models}]{Introduction to Network Models}
\subtitle{Lecture 11}
\author{Agostino Merico}
\date{4 Decemebr 2017}


\begin{document}
\frame{\maketitle}


% \begin{frame}
% \frametitle{Content}
% \large
% \begin{enumerate}
% \item {\color{blue}Equilibria \& stability analysis in one-variable models}\\[18pt]

% \item {\color{blue}Equilibria \& stability analysis in multi-variable
%     models}\\[18pt]

% \item {\color{blue}Sensitivity analysis}\\[18pt]

% \item<2> {\color{red}Graded assignment (final)}
% \end{enumerate}
% \end{frame}



\begin{frame}
\frametitle{Network science -- Basics}
\begin{minipage}{11cm}
\visible<1->{We are now moving into one of the most recent
  developments of complex systems science: \textit{\tr{networks}}.}\\

\visible<2->{Stimulated by two seminal papers on small-world and
  scale-free networks\\ 
\vspace{-0.25cm}

\begin{scriptsize}
\begin{itemize}
\item[-] D. J. Watts \& S. H. Strogatz, Collective dynamics of 'small-world'
networks, \textit{Nature}, 393:440--442, 1998.\\

\item[-] A.-L. Barab\'{a}si \& R. Albert, Emergence of scaling in random
networks, \textit{Science}, 286:509--512, 1999.\\
\end{itemize}
\end{scriptsize}}
\vspace{0.15cm}

\visible<3>{\textit{network science} has been
  rapidly growing and producing novel perspectives, research
  questions, and analytical tools to study various kinds of systems in
  a number of disciplines, including biology, ecology, sociology,
  economics, political science, management science, engineering,
  medicine, and more\,!
}
\end{minipage}
\end{frame}



\begin{frame}
\frametitle{Network science -- Historical roots}
\small
\begin{minipage}{11cm}
\visible<1->{ Historical roots can be found in several disciplines:}\\

\visible<2->{ Mathematics: \textit{graph theory}, the study various properties of
abstract structures called \textit{graphs} made of \textit{nodes}
(a.k.a. vertices) and \textit{edges} (a.k.a. links, ties);}\\

\visible<3->{Physics: \textit{statistical physics}, the study of collective systems
made of a large number of entities (such as phase transitions);}\\

\visible<4->{Social sciences: \textit{Social network analysis};}\\

\visible<5->{Computer science: \textit{Artificial neural networks}.}\\

\visible<6>{Investigations based on network science focus on the
connections and interactions among the components of a system, not
just on each individual component.}
\end{minipage}

\end{frame}


\begin{frame}
\frametitle{Network science -- Terminology}
\small

\begin{minipage}{11cm}
\tb{Network}\\
A \textit{network} (or \textit{graph}) consists of a set of
\textit{nodes} (or \textit{vertices}, \textit{actors}) and a set of
\textit{edges} (or \textit{links}, \textit{ties}) that connect those
nodes.\\

\tb{Neighbour}\\
Node $j$ is called a \textit{neighbour} of node $i$ if (and only if)
node $i$ is connected to node $j$.\\

\tb{Adjacency matrix}\\
A matrix with rows and columns lab
eled by nodes, whose $i$-th row,
$j$-th column component $a_{ij}$ is $1$ if node $i$ is a neighbour of
node $j$, or $0$ otherwise.\\

\tb{Adjacency  list}\\
A list of lists of nodes whose $i$-th component is the list of its
neighbours.

\end{minipage}
\end{frame}


\begin{frame}
\frametitle{Network science -- Example}
\begin{figure}
     \begin{center}
   \includegraphics[width=0.9\textwidth]{images/adj.pdf}\\
\vspace{0.2cm}
\caption*{\footnotesize Examples of an adjacent matrix and an adjacent
  list. The adjacency list offers a more compact, memory-efficient
  representation, especially if the network is sparse (i.e. if the
  network density is low--which is often the case for most rea-world
  networks). But the adjacency matrix also has some
  benefits, such as its feasibility for mathematical analysis and
  easy to access its specific components.}
   \end{center}
\end{figure}

\end{frame}



\begin{frame}
\frametitle{Network science -- More terminology}
\footnotesize
\begin{minipage}{11cm}
\tb{Degree}\\
The number of links connected to a node; the degree of node $i$ is written as \texttt{deg($i$)}.\\

\tb{Walk}\\
A list of links that are sequentially connected to form a continuous route on a
network. In particular: \textit{trail} is a walk that doesn't go thorugh any link more than
  once; \textit{path} is a walk that doesn't go through any node (and
  any link) more than once; \textit{cycle} is a walk that starts and ends at the same node without going
  through any node more than once on its way.\\

\tb{Subgraph}\\
Part of the graph\\

\tb{Connected graph}\\
A graph in which a path exists between any pair of nodes.\\

\tb{Connected component}\\
A subgraph of a graph that is connected within itself but not
connected to the rest of the graph.
\end{minipage}
\end{frame}

\begin{frame}
\frametitle{Network science -- Exercise}
\footnotesize
\vspace{-0.25cm}
\begin{figure}
     \begin{center}
   \includegraphics[width=0.6\textwidth]{images/net1.pdf}
   \end{center}
\end{figure}
\vspace{-0.15cm}

\begin{minipage}{11cm}
1. Represent the network in (a) and adjacency matrix and (b) an adjacency list;\\
\vspace{-0.2cm}

2. Determine the degree for each node;\\
\vspace{-0.2cm}

3. Classify the following walks as trail, path, cycle, or other.\\[-12pt]
\begin{itemize}
\item 6 $\rightarrow$ 3 $\rightarrow$ 2 $\rightarrow$ 4 $\rightarrow$ 2 $\rightarrow$ 1\\[-5pt]
\item 1 $\rightarrow$ 4 $\rightarrow$ 6 $\rightarrow$ 3 $\rightarrow$ 2\\[-5pt]
\item 5 $\rightarrow$ 1 $\rightarrow$ 2 $\rightarrow$ 3 $\rightarrow$ 5\\
\end{itemize}
\vspace{0.05cm}

4. Identify all fully connected three-node subgraphs (i.e. triangles).
\end{minipage}
\end{frame}


\begin{frame}
\frametitle{Network science -- Exercise}
\footnotesize
\begin{minipage}{11cm}

\end{minipage}
\end{frame}



\begin{frame}
\frametitle{Network science -- More terminology}
\footnotesize
\begin{minipage}{11cm}
\tb{Complete graph}\\
A graph in which any pair of nodes are connected;\\

\tb{Regular graph}\\
A graph in which all nodes have the same degree; every
complete graph is regular;\\

\tb{Bipartite ($n$-partite) graph}\\
A graph whose nodes can be divided into two (or $n$) groups so that no
edge connects nodes within each group:\\

\tb{Tree graph}\\
A graph in which there is no cycle; a graph made of multiple trees is
called a forest graph; every tree or forest graph is bipartite;\\

\tb{Planar graph}\\
A graph that can be graphically drawn in a two-dimensional plane with
no edge crossings; every tree or forest graph is planar.
\end{minipage}
\end{frame}


\begin{frame}
\frametitle{Network science -- More terminology}
\begin{columns}
\column{0.2cm}
\column{6cm}
\vspace{-0.5cm}
\begin{figure}
     \begin{center}
   \includegraphics[width=1.1\textwidth]{images/nets.pdf}
   \end{center}
\end{figure}
\column{1cm}
\column{3.5cm}
\scriptsize 
A: Complete graphs \\[16pt]

B: Regular graphs \\[16pt]

C: Bipartite graphs\\[4pt]
~~~~(colours show groups)\\[16pt]

D: Tree graphs\\[16pt]

E: Planar graphs

\end{columns}
\end{frame}



\begin{frame}
\frametitle{Network science -- More terminology}
\footnotesize
\begin{minipage}{11cm}
Classifications of networks according to the types of their links (1
of 3).\\

\tb{Undirected link}\\
 A symmetric connection between nodes; if node $i$ is
connected to node $j$ by an undirected link, then node $j$ also recognises
node $i$ as its neighbor; a graph made of undirected linke is called an
undirected graph; the adjacency matrix of an undirected graph is
always symmetric. \\

\tb{Directed link}\\ 
An asymmetric connection from one node to another; even
if node $i$ is connected to node $j$ by a directed link, the connection
isn't necessarily reciprocated from node $j$ to node $i$; a graph made of
directed links is called a directed graph; the adjacency matrix of a
directed graph is generally asymmetric.\\

\tb{Unweighted link}\\
A link without any weight value associated to
it; there are only two possibilities between a pair of nodes in a
network with unweighted links; whether there is a link between them
or not; the adjacency matrix of such a network is made of only 0s and
1s.
\end{minipage}
\end{frame}


\begin{frame}
\frametitle{Network science -- More terminology}
\footnotesize
\begin{minipage}{11cm}
Classifications of networks according to the types of their links  (2
of 3).\\

\tb{Weighted link}\\ 
A link with a weight value associated to it; a weight
is usually given by a non-negative real number, which may represent
a connection strength or distance between nodes, depending on the
nature of the system being modeled; the definition of the adjacency
matrix can be extended to contain those link weight values for
networks with weighted links; the sum of the weights of links
connected to a node is often called the node strength, which
corresponds to a node degree for unweighted graphs.\\

\tb{Multiple links}\\ 
Links that share the same origin and destination; such
multiple links connect two nodes more than once.\\

\tb{Self-loop}\\ 
A link that originates and ends at the same node.
\end{minipage}
\end{frame}


\begin{frame}
\frametitle{Network science -- More terminology}
\footnotesize
\begin{minipage}{11cm}
Classifications of networks according to the types of their links  (3
of 3).\\

\tb{Simple graph}\\
A graph that does not contain directed, weighted, or
multiple links, or self-loops; traditional graph theory mostly focuses
on simple graphs.\\

\tb{Multigraph}\\
A graph that may contain multiple links; many
mathematicians also allow multigraphs to contain
self-loops; multigraphs can be undirected or directed.\\
\vspace{0.75cm}

\visible<2>{\textbf{NOTE}: According to these taxonomies, all the examples shown in the
previous figure are simple graphs; but many real-world networks are
 modeled using directed, weighted, and/or multiple links.}

\end{minipage}
\end{frame}



\begin{frame}
\frametitle{Constructing network models with \texttt{NetworkX}}
\small
\begin{minipage}{11cm}
Now that we have completed a crash course on graph 
terminology, it is time to begin with computational modelling of
networks.\\

There is a wonderful python module called \texttt{NetworkX} for network modelling
and analysis; it is a free and widely used toolkit.\\

If you use Anaconda, \texttt{NetworkX} is already
installed; if you use Enthought Canopy, you can easily
install \texttt{NetworkX} by using the package manager.\\

The documentation for \texttt{NetworkX} is available online at\\
\url{http://networkx.github.io}.
\end{minipage}
\end{frame}


\begin{frame}
\frametitle{Constructing network models with \texttt{NetworkX}}
\footnotesize
\begin{minipage}{11cm}
Data structure used in \texttt{NetworkX}.\\

\textbf{Graph} For undirected simple graphs (self-loops are allowed);\\

\textbf{DiGraph} For directed simple graphs (self-loops are allowed);\\

\textbf{MultiGraph} For undirected multigraphs (self-loops and
multiple links are allowed);\\ 

\textbf{MultiDiGraph} For directed multigraphs (self-loops and
multiple links are allowed).\\
\vspace{0.5cm}

You can choose one of these four data types that suits your modeling
purposes. We will use mostly Graph and DiGraph.\\

You can construct a graph of your own manually.

\end{minipage}
\end{frame}



\begin{frame}
\frametitle{Constructing network models with \texttt{NetworkX}}
\scriptsize
\vspace{-0.4cm}

\begin{block}{}
\hspace{-0.3cm}\texttt{{\color{blue}import} networkx {\color{blue}as} nx}\\
\vspace{0.15cm}

\visible<2->{
\hspace{-0.3cm}\texttt{{\color{red}\# creates a new empty Graph object}}\\
\hspace{-0.45cm}\texttt{ g = nx.Graph(\,)}\\
\vspace{0.15cm}}

\visible<3->{
\hspace{-0.3cm}\texttt{{\color{red}\# add a node named 'John'}}\\
\hspace{-0.45cm}\texttt{ g.add\_node('John')}\\
\vspace{0.15cm}}

\visible<4->{
\hspace{-0.3cm}\texttt{{\color{red}\# add a bunch of nodes at once}}\\
\hspace{-0.45cm}\texttt{ g.add\_nodes\_from(['Josh', 'Jane', 'Jesse', 'Jack'])}\\
\vspace{0.15cm}}

\visible<5->{
\hspace{-0.3cm}\texttt{{\color{red}\# add a link between 'John' and 'Jane'}}\\
\hspace{-0.45cm}\texttt{ g.add\_edge('John', 'Jane')}
\vspace{0.15cm}}

\visible<6->{
\hspace{-0.3cm}\texttt{{\color{red}\# add a bunch of links at once}}\\
\hspace{-0.45cm}\texttt{ g.add\_edges\_from([('Jesse', 'Josh'), ('John', 'Jack'), ('Jack', 'Jane')])}\\
\vspace{0.15cm}}

\visible<7->{
\hspace{-0.3cm}\texttt{{\color{red}\# add more links}}\\
\hspace{-0.3cm}\texttt{{\color{red}\# undefined nodes will be created automatically}}\\
\hspace{-0.45cm}\texttt{ g.add\_edges\_from([('Jesse', 'Jill'), ('Jill', 'Jeff'), ('Jeff', 'Jane')])}\\
\vspace{0.15cm}}

\visible<8->{
\hspace{-0.3cm}\texttt{{\color{red}\# remove the link between 'John' and 'Jane'}}\\
\hspace{-0.45cm}\texttt{ g.remove\_edge('John', 'Jane')}\\
\vspace{0.15cm}}

\visible<9>{
\hspace{-0.3cm}\texttt{{\color{red}\# remove the node 'John'}}\\
\hspace{-0.3cm}\texttt{{\color{red}\# all links connected to that node will be removed too}}\\
\hspace{-0.45cm}\texttt{ g.remove\_node('John')}}

\end{block}

\end{frame}



\begin{frame}[fragile]
	\forloop{ct}{1}{\value{ct} < 7}
	{
		\only<\value{ct}>{
			\lstinputlisting[language=python, firstline=1, lastline=\value{ct}]{test.py}
		}
	}
\end{frame}



\begin{frame}
\frametitle{Constructing network models with \texttt{NetworkX}}
\vspace{-0.4cm}
\begin{scriptsize}

\begin{minipage}{11cm}
In those examples, I used strings for the names of the nodes, but a
node's name can be a number, a string, a list, a tuple, a dictionary,
etc. \\
\vspace{-0.2cm}

If you execute those command lines, you can see the following
results:\\
\end{minipage}
\vspace{-0.5cm}

\begin{block}{}
\begin{tiny}
\visible<2->{
\hspace{0.5cm}\texttt{$>>>$ g}\\
\hspace{0.5cm}\texttt{<networkx.classes.graph.Graph object at 0x1819cdfe10>}}\\
\vspace{0.25cm}

\visible<3->{
\hspace{0.5cm}\texttt{$>>>$ print(g.nodes)}\\
\hspace{0.5cm}\texttt{['John', 'Josh', 'Jane', 'Jesse', 'Jack',
  'Jill', 'Jeff']}}\\
\vspace{0.25cm}

\visible<4->{
\hspace{0.5cm}\texttt{$>>>$ print(g.edges)}\\
\hspace{0.5cm}\texttt{[('Jane', 'Jeff'), ('Jesse', 'Jill'), ('Jill', 'Jeff')]}}\\
\vspace{0.25cm}

\visible<5>{
\hspace{0.5cm}\texttt{$>>>$ print(g.adj)}\\
\hspace{0.5cm}\texttt{\{'John': \{\,\}, 'Josh': \{\,\}, 'Jane': \{'Jeff':
  \{\,\}\}, 'Jess': \{'Jill': \{\,\}\}, 'Jack': \{\,\},\\ 
\hspace{0.5cm} 'Jill': \{'Jess': \{\,\}, 'Jeff': \{\,\}\}, 'Jeff': \{'Jill': \{\,\}, 'Jane': \{\,\}\}\}}}\\
\end{tiny}
\end{block}
\vspace{0.4cm}

\begin{minipage}{11cm}
\visible<2->{The first output (\texttt{<networkx.classes. ... >}) shows that
\texttt{g} is a Graph object;}\\
\vspace{-0.2cm}

\visible<3->{\texttt{print(g.nodes)} returns a list
of all nodes in the network;}\\ 
\vspace{-0.2cm}

\visible<4->{ \texttt{print(g.edges)} returns a list of all
links};\\ 
\vspace{-0.2cm}

\visible<5>{\texttt{print(g.adj)} provides a view of the adjacency
 list data structure (the links of each node) by the dictionary-like object; \textbf{note}: the
  property of each node are also stored (they are initially empty).}
\end{minipage}

\end{scriptsize}
\end{frame}


\begin{frame}
\frametitle{Constructing network models with \texttt{NetworkX}}
\begin{small}

\begin{minipage}{11cm}
Any node property can be dynamically added or modified as follows: \\
\end{minipage}

\begin{block}{}
\begin{footnotesize}
\hspace{0.5cm}\texttt{$>>>$ g.node['Jeff']['job'] = 'student'}\\
\hspace{0.5cm}\texttt{$>>>$ g.node['Jeff']['age'] = 20}\\
\hspace{0.5cm}\texttt{$>>>$ g.node['Jeff']}\\
\hspace{0.5cm}\texttt{\{'age': 20, 'job': 'student'\}}
\end{footnotesize}
\end{block}

\end{small}
\end{frame}


\begin{frame}
\frametitle{Constructing network models with \texttt{NetworkX}}

\begin{minipage}{11cm}
\begin{small}
In the examples above, we manually constructed network models by
adding or removing nodes and links. But \texttt{NetworkX} also has some
built-in functions that can generate networks of specific shapes.\\
\end{small}
\end{minipage}
\vspace{-0.5cm}

\begin{block}{}
\begin{scriptsize}
\hspace{0.5cm}\texttt{{\color{blue}import} networkx {\color{blue}as} nx}\\
\vspace{0.25cm}

\hspace{0.5cm}\texttt{{\color{red}\# complete graph made of 5 nodes}}\\
\hspace{0.5cm}\texttt{g1 = nx.complete\_graph(5)}\\
\vspace{0.25cm}

\hspace{0.5cm}\texttt{{\color{red}\# complete (fully connected) bipartite graph}}\\
\hspace{0.5cm}\texttt{{\color{red}\# made of group of 3 nodes and group of 4 nodes}}\\
\hspace{0.5cm}\texttt{g2 = nx.complete\_bipartite\_graph(3, 4)}\\
\vspace{0.25cm}

\hspace{0.5cm}\texttt{{\color{red}\# Zachary's Karate Club graph}}\\
\hspace{0.5cm}\texttt{g3 = nx.karate\_club\_graph(\,)}

\end{scriptsize}
\end{block}
\vspace{0.5cm}

\begin{minipage}{11cm}
\begin{small}
The last example (Zachary's Karate Club graph) is a famous classic
example of social networks reported by Wayne Zachary in the '70s;
it is a network of friendships among 34 members of a karate club
at a U.S. university.\\
\end{small}
\end{minipage}

\end{frame}


\begin{frame}
\frametitle{Constructing network models with \texttt{NetworkX}}
\begin{minipage}{11cm}
\begin{small}
The lists of links of the networks above are as follows:\\
\end{small}
\end{minipage}

\begin{block}{}
\begin{tiny}
\hspace{0.5cm}\texttt{$>>>$ print(g1.edges)}\\
\hspace{0.5cm}\texttt{[(0, 1), (0, 2), (0, 3), (0, 4), (1, 2), (1, 3), (1, 4), (2, 3), (2, 4),
(3, 4)]}\\
\vspace{0.35cm}

\hspace{0.5cm}\texttt{$>>>$ print(g2.edges)}\\
\hspace{0.5cm}\texttt{[(0, 3), (0, 4), (0, 5), (0, 6), (1, 3), (1, 4), (1, 5), (1, 6), (2, 3), (2, 4),}\\
\hspace{0.5cm}\texttt{(2, 5), (2, 6)]}\\
\vspace{0.35cm}

\hspace{0.5cm}\texttt{$>>>$ print(g3.edges)}\\
\hspace{0.5cm}\texttt{[(0, 1), (0, 2), (0, 3), (0, 4), (0, 5), (0, 6),
  (0, 7), (0, 8), (0, 10), (0, 11),}\\
\hspace{0.5cm}\texttt{(0, 12), (0, 13), (0, 17), (0, 19), (0, 21), (0, 31), (1, 2),
(1, 3), (1, 7), (1, 13),}\\ 
\hspace{0.5cm}\texttt{(1, 17), (1, 19), (1, 21), (1, 30), (2, 3),
(2, 32), (2, 7), (2, 8), (2, 9), (2, 13),}\\ 
\hspace{0.5cm}\texttt{(2, 27), (2, 28), (3, 7),
(3, 12), (3, 13), (4, 10), (4, 6),(5, 16), (5, 10), (5, 6),}\\
\hspace{0.5cm}\texttt{(6, 16),
(8, 32), (8, 30), (8, 33), (9, 33), (13, 33), (14, 32), (14, 33),
(15, 32),}\\ 
\hspace{0.5cm}\texttt{(15, 33), (18, 32), (18, 33), (19, 33), (20, 32), (20, 33),
(22, 32), (22, 33), (23, 32),}\\ 
\hspace{0.5cm}\texttt{(23, 25), (23, 27), (23, 29), (23, 33),
(24, 25), (24, 27), (24, 31), (25, 31), (26, 33),}\\ 
\hspace{0.5cm}\texttt{(26, 29), (27, 33),
(28, 33), (28, 31), (29, 32), (29, 33), (30, 33), (30, 32), (31,
33),}\\
\hspace{0.5cm}\texttt{(31, 32), (32, 33)]}

\end{tiny}
\end{block}

\end{frame}



\begin{frame}
\frametitle{Visualising networks with \texttt{NetworkX}}
\begin{minipage}{11cm}
\begin{small}
NetworkX also provides functions for visualizing networks; they are
not as powerful as other more specialised software, but still quite
handy and useful, especially for small- to mid-sized network
visualization; those visualization functions depend on the functions
defined in matplotlib (pylab), so we need to import it before
visualizing networks.\\

The simplest way is to use the NetworkX's \texttt{draw} function, as follows:\\
\end{small}
\end{minipage}

\begin{block}{}
\begin{scriptsize}
\hspace{0.5cm}\texttt{{\color{blue}from} pylab {\color{blue}import} *}\\
\hspace{0.5cm}\texttt{{\color{blue}import} networkx {\color{blue}as} nx}\\
\vspace{0.25cm}

\hspace{0.5cm}\texttt{g = nx.karate\_club\_graph(\,)}\\
\hspace{0.5cm}\texttt{nx.draw(g)}\\
\hspace{0.5cm}\texttt{show(\,)}

\end{scriptsize}
\end{block}

\end{frame}



\begin{frame}
\frametitle{Visualising networks with \texttt{NetworkX}}
\begin{minipage}{11cm}
\begin{small}
The layout of the nodes and links is automatically determined by the
\textit{Fruchterman-Reingold force-directed} algorithm (called "spring
layout" in NetworkX); this heuristic algorithm tends to bring
groups of well-connected nodes closer to each other, making the result
of visualization more meaningful and aesthetically more pleasing.\\
\end{small}
\end{minipage}

\begin{figure}
     \begin{center}
   \includegraphics[width=0.5\textwidth]{images/karate.png}\\
\vspace{-0.1cm}
\caption*{\tiny Visual output of the Zachary's Karate Club
  network. Your result may not look like this because the spring
  layout algorithm uses random initial positions.}
   \end{center}
\end{figure}
\end{frame}



\begin{frame}
\frametitle{Visualising networks with \texttt{NetworkX}}
\begin{minipage}{11cm}
\begin{small}
There are several other layout algorithms; also, there are many options you can use to customise
visualization results; check out the NetworkX's online documentation
to learn more about what you can do. For example:\\
\end{small}
\end{minipage}
\vspace{-1cm}

\begin{block}{}
\begin{tiny}
\hspace{0.5cm}\texttt{{\color{blue}import} pylab {\color{blue}as} pl}\\
\hspace{0.5cm}\texttt{{\color{blue}import} networkx {\color{blue}as} nx}\\
\vspace{0.2cm}

\hspace{0.5cm}\texttt{g = nx.karate\_club\_graph(\,)}\\
\hspace{0.5cm}\texttt{pl.subplot(2,2,1)}\\
\hspace{0.5cm}\texttt{nx.draw\_random(g, node\_size=60)}\\
\hspace{0.5cm}\texttt{pl.title('random layout)}\\
\vspace{0.2cm}

\hspace{0.5cm}\texttt{pl.subplot(2, 2, 2)}\\
\hspace{0.5cm}\texttt{nx.draw\_circular(g, node\_size=60)}\\
\hspace{0.5cm}\texttt{pl.title('circular layout')}\\
\vspace{0.2cm}

\hspace{0.5cm}\texttt{pl.subplot(2, 2, 3)}\\
\hspace{0.5cm}\texttt{nx.draw\_spectral(g, node\_size=60)}\\
\hspace{0.5cm}\texttt{pl.title('spectral layout')}\\
\vspace{0.2cm}

\hspace{0.5cm}\texttt{pl.subplot(2, 2, 4)}\\
\hspace{0.5cm}\texttt{shells = [[0, 1, 2, 32, 33],}\\
\hspace{0.75cm}\texttt{[3, 5, 6, 7, 8, 13, 23, 27, 29, 30, 31],}\\
\hspace{0.75cm}\texttt{[4, 9, 10, 11, 12, 14, 15, 16, 17, 18,}\\
\hspace{0.75cm}\texttt{19, 20, 21, 22, 24, 25, 26, 28]]}\\
\hspace{0.5cm}\texttt{nx.draw\_shell(g, , node\_size=60, nlist = shells)}\\
\hspace{0.5cm}\texttt{pl.title('shell layout')}\\
\hspace{0.5cm}\texttt{pl.show(\,)}

\end{tiny}
\end{block}
\end{frame}




\begin{frame}
\frametitle{Visualising networks with \texttt{NetworkX}}
\begin{figure}
     \begin{center}
   \includegraphics[width=0.8\textwidth]{images/layout.png}\\
\vspace{-0.1cm}
\caption*{\small Visual output showing examples of network layouts available in NetworkX.}
   \end{center}
\end{figure}
\end{frame}



\begin{frame}
\frametitle{Visualising networks with \texttt{NetworkX}}
\vspace{-0.7cm}
\begin{block}{}
\begin{tiny}
\hspace{0.5cm}\texttt{{\color{blue}import} pylab {\color{blue}as} pl}\\
\hspace{0.5cm}\texttt{{\color{blue}import} networkx {\color{blue}as} nx}\\
\vspace{0.2cm}

\hspace{0.5cm}\texttt{g = nx.karate\_club\_graph(\,)}\\
\vspace{0.2cm}

\hspace{0.5cm}\texttt{positions = nx.spring\_layout(g)}\\
\vspace{0.2cm}

\hspace{0.5cm}\texttt{pl.subplot(3, 2, 1)}\\
\hspace{0.5cm}\texttt{nx.draw(g, positions, node\_size=180, with\_labels=True)}\\
\hspace{0.5cm}\texttt{pl.title('showing node names')}\\
\vspace{0.2cm}

\hspace{0.5cm}\texttt{pl.subplot(3, 2, 2)}\\
\hspace{0.5cm}\texttt{nx.draw(g, positions, node\_size=60, node\_shape='>')}\\
\hspace{0.5cm}\texttt{pl.title('using different node shape')}\\
\vspace{0.2cm}

\hspace{0.5cm}\texttt{pl.subplot(3, 2, 3)}\\
\hspace{0.5cm}\texttt{nx.draw(g, positions, node\_size=[g.degree(i)*50 {\color{blue}for} i {\color{blue}in} g.nodes()])}\\
\hspace{0.5cm}\texttt{pl.title('changing node sizes')}\\
\vspace{0.2cm}

\hspace{0.5cm}\texttt{pl.subplot(3, 2, 4)}\\
\hspace{0.5cm}\texttt{nx.draw(g, positions, node\_size=60, edge\_color='pink',}\\ 
\hspace{1.5cm}\texttt{node\_color=['yellow' {\color{blue}if} i$<$17 {\color{blue}else} 'green' {\color{blue}for} i {\color{blue}in} g.nodes(\,)]) }\\
\hspace{0.5cm}\texttt{pl.title('coloring nodes and links')}\\
\vspace{0.2cm}

\hspace{0.5cm}\texttt{pl.subplot(3, 2, 5)}\\
\hspace{0.5cm}\texttt{nx.draw\_networkx\_nodes(g, positions, node\_size=60)}\\
\hspace{0.5cm}\texttt{pl.title('nodes only')}\\
\vspace{0.2cm}

\hspace{0.5cm}\texttt{pl.subplot(3, 2, 6)}\\
\hspace{0.5cm}\texttt{nx.draw\_networkx\_edges(g, positions)}\\
\hspace{0.5cm}\texttt{pl.title('links only')}\\
\vspace{0.2cm}

\hspace{0.5cm}\texttt{pl.show(\,)}\\

\end{tiny}
\end{block}
\end{frame}


\begin{frame}
\frametitle{Visualising networks with \texttt{NetworkX}}
\vspace{-0.7cm}
\begin{figure}
     \begin{center}
   \includegraphics[height=0.65\textwidth]{images/layout-options.png}\\
\vspace{-0.25cm}
\caption*{\tiny Examples of
  drawing options available in NetworkX; the same node positions are
  used in all panels; the last two examples show the axes because
  they are generated using different drawing functions; to suppress
  the axes, use \texttt{axis('off')} right after the network drawing.}
   \end{center}
\end{figure}
\end{frame}




\begin{frame}
\frametitle{Dynamical network models}
\begin{minipage}{11cm}
\begin{scriptsize}

There are different categories of dynamical network models. We will
look at the following three:\\

\tb{Models for "dynamics \textit{on} networks''}.\\ 
These models consider how the states of components (i.e. the nodes)
change over time through their interactions with other nodes that are
connected to them. Network topology is fixed throughout time. Cellular
automata, boolean networks, and artificial neural networks (without
learning) all belong to this class.\\

\tb{Models for "dynamics \textit{of} networks"}.\\ 
These are the models that consider
changes in network topology, for various
purposes: to understand mechanisms that bring particular network
topologies, to evaluate robustness and vulnerability of networks, to
design procedures for improving certain properties of networks,
etc. \\

\tb{Models for "adaptive networks"}.\\ 
These are models that describe the co-evolution
of dynamics on and of networks, where node states and network
topologies dynamically change . Adaptive
network models
provide a generalised modeling framework for complex systems.

\end{scriptsize}
\end{minipage}
\end{frame}

\begin{frame}
\frametitle{Dynamical network models}
\begin{minipage}{11cm}
\begin{footnotesize}
Because NetworkX adopts dictionaries as main data
structure, we can easily add states to nodes (and links) and
dynamically update those states iteratively. This is a simulation of
dynamics on networks.\\

Many real-world dynamical networks fall into this category,
including:\\
\vspace{-0.25cm}

\begin{itemize}
\item Regulatory relationships among genes and proteins within a cell,
  where nodes are genes and/or proteins and the node states are their
  expression levels.\\[6pt]
\item Ecological interactions among species in an ecosystem, where
  nodes are species and the node states are their populations.\\[6pt]
\item Disease infection on social networks, where nodes are
  individuals and the node states are their epidemiological states
  (e.g., susceptible, infected, recovered, immunised, etc.).\\[6pt]
\item Information/culture propagation on organizational/social
  networks, where nodes are individuals or communities and the node
  states are their informational/cultural states.
\end{itemize}

\end{footnotesize}
\end{minipage}
\end{frame}


%-------------------------------------------------------------




\begin{frame}
  \frametitle{Further reading}

  \begin{thebibliography}{10}
    
    \scriptsize

    \beamertemplatebookbibitems % block for books

  \bibitem[Sayama, 2015]{SAYA15}
    Hiroki Sayama~2015
    \newblock {\em Introduction to the Modeling and Analysis of
      Complex Systems}
    \newblock Open SUNY Textbooks 

 
    \beamertemplatearticlebibitems % block for articles
 

    \beamertemplateonlinebibitems % block for online material

   \bibitem[NetworkX 2.0, 2017]{NX2.0}
   NetworkX Reference 2.0
    \newblock \url{https://networkx.github.io/documentation/stable/_downloads/networkx_reference.pdf}
    \newblock Sep 20, 2017



  \end{thebibliography}

\end{frame}





\end{document}