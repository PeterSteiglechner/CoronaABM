\documentclass[x11names,svgnames,table]{beamer}
\usetheme{sharky}

\def\Put(#1,#2)#3{\leavevmode\makebox(0,0){\put(#1,#2){#3}}}

\usepackage{calc}

\usepackage[compatibility=false, justification=justified]{caption}    % to allow the use of \caption* i.e. caption without numbering

\usepackage{amssymb}
\usepackage[T1]{fontenc}

%\usepackage[scaled]{beramono} % allows the use of textbf under texttt

\usepackage{bbding} % for symbols such as \HandCuffRight

\usepackage{eqnarray,amsmath}
\usepackage{mathtools}

\usepackage{forloop}
\newcounter{ct} 
\newcounter{ll}
%\usepackage[table]{xcolor}  % to color rows of tables using \rowcolor{color} before the table row. 
                                                  % colors include red, blue, black, gray, yellow and many other popular colors.

\usepackage{ragged2e} % to justify itemized text

\usepackage{media9}   % this to embed youtube video in the pdf as flash file 
\usepackage{movie15} % gif
\usepackage{animate}  % to created animated png


% defining new commands
\newcommand\bhline{\arrayrulecolor{blue}\hline\arrayrulecolor{black}}
\newcommand\tg{\textcolor[gray]{0.8}}
\newcommand\tgr{\textcolor{green}}
\newcommand\tog{\textcolor{OliveGreen}}
\newcommand\tpg{\textcolor{PineGreen}}
\newcommand\tgo{\textcolor{Goldenrod}}
\newcommand\tro{\textcolor{RedOrange}}
\newcommand\tyo{\textcolor{YellowOrange}}
\newcommand\tbo{\textcolor{BurntOrange}}
\newcommand\ty{\textcolor{yellow}}
\newcommand\tr{\textcolor{red}}
\newcommand\tb{\textcolor{blue}}

\newcommand\bi{\begin{itemize}}
\newcommand\ei{\end{itemize}}
\newcommand\be{\begin{enumerate}}
\newcommand\ee{\end{enumerate}}
\newcommand\bce{\begin{center}}
\newcommand\ece{\end{center}}
\newcommand\beq{\begin{equation}}
\newcommand\eeq{\end{equation}}
\newcommand\bea{\begin{eqnarray}}
\newcommand\eea{\end{eqnarray}}

\newcommand\bl{\begin{lstlisting}}
\newcommand\el{\end{lstlisting}}
	
\newcommand\bfig{\begin{figure}}
\newcommand\efig{\end{figure}}

\usepackage{color}
\definecolor{mygreen}{rgb}{0,0.6,0}
\definecolor{mygray}{rgb}{0.5,0.5,0.5}
\definecolor{mymauve}{rgb}{0.58,0,0.82}
\definecolor{myblue}{rgb}{0.0, 0.18, 0.65}

\usepackage{listings} % for displaying programming code in LaTex (see: https://www.sharelatex.com/learn/Code_listing) 

    \lstset{basicstyle=\tiny\ttfamily,  % size and style of fonts used for the code
      backgroundcolor=\color{white},  % choose the background color
      breaklines=true,                          % break lines when too long 
      resetmargins=true,                   % reset the margins
      xleftmargin=-0.45cm,                 % set the left margin
      xrightmargin=0.0cm,                   % set the right margin
      showspaces=false,                       % show spaces adding particular underscores
      showstringspaces=false,             % underline spaces within strings
      showtabs=false,                          % show tabs within strings adding 
      escapechar=@,                             % to add LaTeX commands within the code
      keywordstyle=\color{mymauve},  % keyword style
      commentstyle=\color{red},           % comment style
      %stringstyle=\color{mygreen}       % string literal style
      tabsize=8,
      language=python,
      numberstyle=\tiny
} 

\usepackage{hyperref}
\hypersetup{
    colorlinks=true,
    linkcolor=blue,
    filecolor=blue,      
    urlcolor=blue,
}
\urlstyle{same}

\usepackage{changepage}
\usepackage{bold-extra}

\input{/home/peter/Organisatorisches/Latex_dinge/packages_beamer}

% TABULARX
\usepackage{tabularx}
\newcolumntype{L}[1]{>{\raggedright\arraybackslash}p{#1}}


\addbibresource{CoronaABM.bib}


\setbeamercolor{block body alerted}{bg=alerted text.fg!10}
\setbeamercolor{block title alerted}{bg=alerted text.fg!20}
\setbeamercolor{block body}{bg=structure!10}
\setbeamercolor{block title}{bg=structure!20}
\setbeamercolor{block body example}{bg=green!10}
\setbeamercolor{block title example}{bg=green!20}
\setbeamertemplate{blocks}[rounded][shadow]


\title[{\color{white}Lecture 12~~--~~Agent-Based Model of Covid-19}]{Agent-Based Model of Covid-19} % An Agent-Based Model for SARS-CoV-2 transmission and containment policies
\subtitle{Lecture 12}
\author{Peter Steiglechner}
\date{26 November 2020}

\begin{document}
\frame{\maketitle}

\begin{frame}<1-3>[label=Outline]{What we will cover today.}
	\setbeamercovered{transparent}
	\begin{itemize}%[<+-|alert@+>]
	\item<1> An Agent-Based Model of the spread of Covid-19 in a small society in order to test local policies
	\item<2> How to make agents heterogeneous: Drawing from probability distributions
	\item<3> How to let agents interact with each other: Social Networks for the interaction of agents
\end{itemize}

\end{frame}


\begin{frame}{Coronavirus and Covid-19}
	% covid big topic.
	\bi 
	\item Appearance of a new virus `Severe Acute Respiratory Syndrome Coronavirus-2 (SARS-CoV-2)' in December 2019 in Wuhan, China.
	\item The virus causes the disease Covid-19
	\item Virus spreads mainly airborne from infectious humans to other humans (aerosols, ...)
	\item Declared pandemic by WHO on 31st January, 2020.
	\item How dangerous is the Covid-19? 
	\bi 
	\item In March strongly increased death rates % (and case fatality rate)
	\item In beginning, large reproductive number, $R_{\rm 0}$
	\item \ra We still don't know that much about Covid-19 and Coronavirus-2
	\ei
	\ei
	Sources e.g.\ \url{https://www.ecdc.europa.eu/en/covid-19/facts/questions-answers-basic-facts}
	% What is it, How does it spread, ...
	
\end{frame}

\begin{frame}{Covid-19: Course of an infection}
\only<1>{
	Latent, Pre-Symptomatic, Symptomatic, or Asymptomatic?
	\begin{itemize}
		\item  Not all infected people develop symptoms (i.e.\ asymptomatic infection), but they might still transmit the virus. This is age dependent (e.g.\ children barely show symptoms).
		\item After a latent phase, the virus can be transmitted in \textbf{every} phase of the infection. 
		{\color{red}In particular, the infection period starts \textbf{before} the onset of symptoms (pre-symptomatic).}
		%{\color{red}{Not all infected people develop symptoms (i.e.\ asymptomatic infection), but they might still transmit the virus. The course of an infection depends on the age (e.g.\ children barely show symptoms). \newline In \textbf{all} stages except for the latent infection the virus can be transmitted: In particular, the infection period starts \textbf{before} the onset of symptoms (pre-symptomatic).}}
		\item Incubation time: \newline 
		{Delay between exposure ("catching the virus") and symptom onset. Ranging between 1 and 14 days (Mean ca.\ 5 days.)}
		%\item Latent Period, Infectious Period: \newline
		%{\color{red}{After exposure, the infected is often latent. However, the infection period typically starts \textit{before} the symptom onset at which it peaks.}}
	\end{itemize}
}
	\begin{center}
		
		\only<2>{\includegraphics[width=\textwidth]{images/courseInfection_Corona1}} 

		\only<3>{\includegraphics[width=\textwidth]{images/courseInfection_Corona2}} 
		
		\only<4>{\includegraphics[width=\textwidth]{images/courseInfection_Corona3}} 
	
		\only<5>{\includegraphics[width=\textwidth]{images/courseInfection_Corona4}} 				
				
		%{\tiny \color{gray}  Meyerowitz, EA, et al.\ (2020) \href{https://doi.org/10.1096/fj.202000919}{Rethinking the role of hydroxychloroquine in the treatment of COVID‐19}. The FASEB Journal.}
	\end{center}

\end{frame}

\begin{frame}{Covid-19: Stats}
	\begin{itemize}
		\item Recovery or Death \newline
		{\color{red}{The Case Fatality Rate $cfr = \frac{\#\ \text{deaths}}{\# \ \text{confirmed cases}}$ depends strongly on age. 
		\newline 
		Note, this is not "how deadly the virus is" (\ra Infection Fatality Ratio $ifr =\frac{\#\ \text{deaths}}{\# \ \text{infected cases}}$) but is often used as a proxy.}}
		\item $R_{0}$-Value \newline
		{\color{red}{How many people catch the virus from one infected person (on average) in a \textbf{fully susceptible society}. Observed $R_0$-values ranged between $1.5 - 2.5$. However, it is assumed that infectiousness varies strongly among different people$^{\rm 1}$.}}
		\item Serial Interval \newline
		{\color{red}{The time between symptom onsets of consecutive infections (person A infected person B): $t_{\rm Symptom, \ B} - t_{\rm Symptom, \ A} \approx 5.8 \ {\rm days}$ \citep{he_temporal_2020}}}
	\end{itemize}
{\tiny (1): Adam, D.\ (2020) \href{https://doi.org/10.1038/d41586-020-02009-w}{“A Guide to R — the Pandemic’s Misunderstood Metric.”} in \textit{Nature}. 
}
\end{frame}
\begin{frame}{Question}
	\begin{framed}
	{\color{red}Why is Covid-19 so difficult to manage and dangerous? \textbf{Your answers}}
	\end{framed}
	
\centering
\includegraphics[width=\linewidth]{Screenshots_Zoom/01_final}

\end{frame}

\begin{frame}{Specifics of Covid-19}
	Why is Covid-19 so difficult to manage and dangerous? 
	\bi 
	\item long delay between start of infectiousness period and symptom onset
	\item Asymptomatic infection (?)
	\item High death rate for a part of the society
	\item Uncertainty!!
	\ei 
	\vfill
	{\scriptsize E.g.\ \citet{he_temporal_2020}}
\end{frame}

\begin{frame}{Policy approaches}
	\bi 
	\item Total Lock-down (e.g.\ Italy. But never perfect. E.g.\ supermarket, police, households)
	\item Strict quarantine for positive cases and first contacts (Germany, spring 2020)
	\item Social distancing 
	\bi
	\item social part: encouraged to reduce contacts (i.e.\ reduce Nr of potential new infected people)
		\bi 
		\item Close schools (reduce number of asymptomatic transmissions)
		\item ...
		\ei 
	\item distancing part: avoid close contacts (i.e.\ reduce chance of transmitting virus)
	\ei 
	\item Herd Immunity
	\item Wear a mask (i.e.\ reduce chance of transmitting virus to random people)
	\ei
	\color{red}{Policies vary in their epidemiological, social and economic implications} %\ra We need to understand these effects}
\end{frame}

\begin{frame}{Role of Modeling - SIR}
	\bi 
	\item Initial projections: Based on $R_0$ estimates/measurements. Became known via `flatten the curve'.
	\item More complex, realistic models emerged (e.g.\ age-structured, which is important for projecting the pressure on the health system)
	\item Hindsight analysis: Which policy strategy was efficient? E.g.\ \citet{reiner_modeling_2020}
	\item Assessing impacts of various political decisions:
			The recent suggestion of `circuit-breaker' again is fueled by mathematical modelling \citep{keeling_precautionary_2020}
	\ei
\end{frame}

\begin{frame}{Role of Modeling -  An ABM shapes national politics}
	\begin{columns}
		\begin{column}{0.5\textwidth}
			\centering
		\includegraphics[width=0.9\linewidth]{images/britishGov_corona} \newline
		{ \href{https://www.theguardian.com/world/2020/mar/16/new-data-new-policy-why-uks-coronavirus-strategy-has-changed}{The Guardian, 16 Mar 2020}}
		\end{column}
	\begin{column}{0.5\textwidth}
			In the beginning of the pandemic, the British government changed its control strategy mainly due to research from Imperial College and in particular one agent-based model exploring the effectiveness of policies w.r.t.\ Covid-19:
		\citet{ferguson_report_2020}.
	\end{column}
	\end{columns}
\end{frame}

\begin{frame}{Different types of Models}
\only<1>{	\bi 
	\item SIR Models
	\bi 
		\item Differential Equations for aggregate variables: Size of the population of susceptible/(exposed)/infected/recovered people
	\ei 
	\item Agent Based Model:
	\bi
		\item List of discrete entities "agents", with rules specifying their individual `trajectories' or behaviour (e.g.\ what they do when infected)
	\ei 
	\ei
	\vfill
	\begin{framed}
	{\color{red}What are advantages/disadvantages of both approaches (in general and related to Covid-19)?}
	\end{framed}
	}
\only<2>{
	\begin{framed}
	{\color{red}What are advantages/disadvantages of both approaches (in general and related to Covid-19)? \textbf{Your answers}}
	\end{framed}
\centering
\includegraphics[width=\linewidth]{Screenshots_Zoom/02_final}
}
\end{frame}
%\todo{I want to encourage them to say things and write them simultaneously on a mindmap. And give hints like: "Imagine I'm a mayor of a city/the WHO. What model is more appropriate for my purposes? (And for what?)."}


\begin{frame}{Why is ABM useful in this pandemic}

	\begin{itemize}
		\item The Virus and Humans act non-homogeneously:
		{
		\bi
			\item {\footnotesize \color{gray} The pandemic is not homogeneously distributed in space}
			 \item {\footnotesize \color{gray} Each person has a different health response to catching the virus }
				%			\ei
				%		}
				%		\item Humans behave non-homogeneous \newline
			\item {\footnotesize \color{gray} 
			Will they quarantine? How many social contacts do they have?
			}
			\ei 
		}
		\item The modeled world is complex and we need to simulate the actual sequence of events.  \newline
		{
			\footnotesize \color{gray}
			If we have two entirely segregated societies, an outbreak of Covid-19 in society A does not impact society B at all. 
		}
		\item Single/Microscopic events can play a crucial role! \newline
		{\footnotesize \color{gray}
			E.g.\ carnival in Heinsberg (Germany). The pandemic wouldn't exist if "patient 0" in Wuhan had somehow avoided all contacts while infectious
		}
		\item Uncertainty and random events play a role: \newline
		{
			\footnotesize \color{gray} Can't predict what consequences an action (like a large social event) has.
		}
		\item The course of the events/decisions changes the `rules of the game' (non-ergodic system) \newline
		{
				\footnotesize \color{gray} Wuhan citizens react differently to a 2nd wave than New Zealanders. 
			}
	\end{itemize}

\end{frame}



\begin{frame}{\textbf{Concept 0 Summary:}}
\begin{block}{Consider ABM when:}
	\bi 
	\item Microscopic behaviour can cause \textbf{emergent} macroscopic phenomena. \newline
	{\color{gray} Example: Fish swarm or bird flock}
	\item People, space, or responses (processes) are \textbf{non-homogeneous} with potentially non-linear feedbacks. \ra We can't reduce the population to a representative agent (\textbf{irreducibility})
	\newline 
	{\color{gray}\textit{Il mondo bello per que se vario.}}
	\item \textbf{Uncertainty} plays a dominant role. 
	\item Context matters (\textbf{non-ergodic system})
	\ei
\end{block}
	{\footnotesize If interested, read the very enlightening book `The end of theory' by \citet{bookstaber_end_2017}}
\end{frame}

\againframe<1>{Outline}


\begin{frame}<1-3>[label=overview]{Overview ABM}
	Let's develop an ABM (Slide 14/56 from Lecture 10 on ABM)
	\be
	\item Specific problem to be solved by the ABM. \newline \only<1>{\small \ \newline \ \newline}
	\only<2->{\small {\color{red} How do a few infected agents affect a small, interconnected, simple society split into three age-/riskgroups? What are the impacts of certain local policies?}}
	\item Design of agents and their static / dynamic attributes.\newline \only<1,2>{\small \ \newline }
	\only<3->{\small {\color{red} Class \textit{Agent()}, function \textit{initialise()}, that creates heterogeneous agents, and helper-function \textit{catch\_virus} that creates an agent's infection course (when the agent catches the virus)}}
	\item \sout{Design of an environment and the way agents interact with it.}
	\item Design of agents’ mutual interactions.\newline
	\only<4->{\small {\color{red} Via social network of agents}}	
	\item Design of agents’ behaviours. \newline
	\only<5->{\small {\color{red} Via function \textit{update()} and for single agent \textit{catch\_virus}}}
	\item Availability of data.	\only<6->{AND}
	\item Method of model validation. \newline \only<1,2,3,4,5>{\small \ \newline }
	\only<6->{\small {\color{red} E.g.\ via the obtained reproductive number $R_0$ (not shown here)}}
	\ee
\end{frame}

     
\begin{frame}
  \frametitle{What do we need to model Covid-19}
        \begin{itemize}
        \item Understand individual infection course \newline  {\color{gray} what happens with infected people} %\visible<2>{{\color{red} \ra Were you infected? Recall your infection course?}}
        \item Understand transmissions \newline {\color{gray} how do people infect each other}
        \item Understand social structure \newline  {\color{gray} who infects who}
        \end{itemize}
      \end{frame}

      \begin{frame}
        \frametitle{Covid-19: Course of an infection (2020)}
        \begin{overlayarea}{0.9\textwidth}{0.8\textheight}
\only<1->{Latent period? Pre-Symptomatic and symptomatic, or asymptomatic infection?}%
\vspace{0.4cm}
\begin{center}
	\only<2>{\includegraphics[width=\textwidth]{images/courseInfection_Corona1}} %
		\only<3>{\includegraphics[width=\textwidth]{images/courseInfection_Corona2}} %
		\only<4>{\includegraphics[width=\textwidth]{images/courseInfection_Corona3}} %
                \only<5->{\includegraphics[width=\textwidth]{images/courseInfection_Corona4}} %
              \end{center}
                \vfill
                \vspace{0.4cm}
                \only<6->{\ra infection course is age-dependent}
		%{\tiny \color{gray}  Meyerowitz, EA, et al.\ (2020) \href{https://doi.org/10.1096/fj.202000919}{Rethinking the role of hydroxychloroquine in the treatment of COVID‐19}. The FASEB Journal.}
              \end{overlayarea}
\end{frame}

\begin{frame}
  \frametitle{Model sketch}
  \begin{columns}
    \begin{column}{0.75\textwidth}
  \centering
  \only<1>{\includegraphics[height=0.5\textheight]{images/model0}}%
  \only<2>{\includegraphics[height=0.5\textheight]{images/model1}}%
  \only<3>{\includegraphics[height=0.5\textheight]{images/model2}}%
  \only<4>{\includegraphics[height=0.5\textheight]{images/model3}}%
  \only<5>{\includegraphics[height=0.5\textheight]{images/model4}}%
  \only<6>{\includegraphics[height=0.5\textheight]{images/model0}}%

  \vfill
  \only<1->{\includegraphics[width=0.8\textwidth]{images/courseInfection_Corona4}} %
\end{column}
\begin{column}{0.2\textwidth}
\only<2->{  simple contagion \newline \ \newline  vs. \newline \  \newline complex contagion}
  
  \end{column}
\end{columns}

\end{frame}


\begin{frame}<1-2>[label=Init, fragile]
\frametitle{Agent and initialisation}
\begin{overlayarea}{\linewidth}{12cm}
\begin{onlyenv}<1>
\begin{lstlisting}[basicstyle=\tiny\ttfamily, tabsize=6] 


	
	
class Agent:
	pass
\end{lstlisting}

  \vspace{0.2cm}

An empty class.
\end{onlyenv}
\begin{onlyenv}<2>
\begin{lstlisting}[basicstyle=\tiny\ttfamily, tabsize=6] 
N_AGENTS = 1000 



class Agent:
	pass

def initialise():
	global agents
	agents = []

	for i in range(N_AGENTS):
		ag = Agent()
		ag.id = i
		ag.health_state = "susceptible"	
		ag.age = ???			
		agents.append(ag)
\end{lstlisting} 
  \vspace{0.2cm}

  Create all agents.
  
  \vspace{0.2cm}
  
\textbf{Problem}: want $20\,\%$ children, $50\,\%$ adults/low-risk, $30\,\%$ elderly/high-risk
\newline
\ 
\end{onlyenv}
\begin{onlyenv}<3>
\begin{lstlisting}[basicstyle=\tiny\ttfamily, tabsize=6] 
N_AGENTS = 1000  
AGE_GROUPS = ["child", "adult", "elderly"] 
FRACTION_AGE_GROUPS = [0.2, 0.5, 0.3]  

class Agent:
	pass

def initialise():
	global agents
	agents = []

	for i in range(N_AGENTS):
		ag = Agent()
		ag.id = i			
		ag.health_state = "susceptible"	
		ag.age = np.random.choice(AGE_GROUPS, p=FRACTION_AGE_GROUPS) 
		agents.append(ag)
\end{lstlisting} 

  \vspace{0.2cm}
  
\textbf{Problem}: want $20\,\%$ children, $50\,\%$ adults/low-risk, $30\,\%$ elderly/high-risk
\newline
\ \Checkmark \ Draw from discrete probability distribution for each agent
\end{onlyenv}
\begin{onlyenv}<4>
\begin{lstlisting}[basicstyle=\tiny\ttfamily, tabsize=6] 
N_AGENTS = 1000
AGE_GROUPS = ["child", "adult", "elderly"] 
FRACTION_AGE_GROUPS = [0.2, 0.5, 0.3]  

class Agent:
	pass

def initialise():
	global agents
	agents = []

	for i in range(N_AGENTS):
		ag = Agent()
		ag.id = i			
		ag.health_state = "susceptible"	 	
		ag.age = np.random.choice(AGE_GROUPS, p=FRACTION_AGE_GROUPS) 
		agents.append(ag)
\end{lstlisting} 

  \vspace{0.2cm}

  \textbf{Problem}: want $20\,\%$ children, $50\,\%$ adults/low-risk, $30\,\%$ elderly/high-risk
\newline
\ \Checkmark \ Draw from discrete probability distribution for each agent \newline
\Put(130,280){\includegraphics[width=0.7\linewidth]{images/DistAge}}
\end{onlyenv}
\begin{onlyenv}<5>
\begin{lstlisting}[basicstyle=\tiny\ttfamily, tabsize=6] 

...
N_INFECTED_INIT = 2  # Number of agents in state "exposed" at t=0.

class Agent:
	pass

def initialise():
	global agents
	agents = []

	for i in range(N_AGENTS):
		ag = Agent()
		ag.id = i 
		ag.health_state = "susceptible"
		ag.age = np.random.choice(AGE_GROUPS, FRAC_AGE_GROUPS) 
		agents.append(ag)

	symptomatic_agents = np.random.choice(agents, size=N_INFECTED_INIT)
	for ag in symptomatic_agents:
		catch_virus(ag, t=0)

	return
\end{lstlisting} 
Infect a few randomly selected agents \newline (we will define function \textit{catch\_virus} in the next slides)
\end{onlyenv}
\end{overlayarea}
\end{frame}

\section{Concept 1: Drawing from discrete distributions}

%\subsection{Concept 1: Discrete Distributions}
\begin{frame}[fragile,t]
\frametitle{Concept 1: Discrete distributions -- a simple example}
\begin{example}
Problem: create a population with an attribute `sex' \newline \ \newline 
\visible<2->{\ \Checkmark \ For each agent, assign sex to male with $50\,\%$ and female else 
\newline}%
\visible<3->{\ \Checkmark \ For each agent, randomly draw one of two sexes (male/female with each $50\,\%$ probability) \newline \ra \textit{{\color{red} np.random.choice(options, size, replace, probability)}} \newline i.e.\ choose \textit{size} samples of \textit{options} with given \textit{probabilities}.}
\end{example}

\pause
\begin{lstlisting}[basicstyle=\tiny\ttfamily, tabsize=6]
for ag in agents:
    ag.sex = "male" if np.random.random() < 0.5 else "female"
\end{lstlisting}


\pause 
\begin{lstlisting}[basicstyle=\tiny\ttfamily, tabsize=6]
for ag in agents:
	ag.sex = np.random.choice(
		["male", "female"] # list/array of possible options
		p =[0.5, 0.5],     # probabilities assigned to these 
		size = 1,          # nr of draws (default=1)
            replace = True     # sample with/without replacement (default=True)
          )
\end{lstlisting}
%}

\end{frame}


\againframe<3->{Init}




\section{Concept 2: Drawing from continuous distributions}

\begin{frame}
  \frametitle{Concept 2: Continuous distributions}
	\begin{block}{Random variable}
	\begin{overlayarea}{\linewidth}{0.8\textheight}
	\only<1->{
	\begin{itemize}
	\item Random variable $x$
	\item Probability density function PDF: $p(x)$
	\item Needs to integrate to one: $\int_{-\infty}^{\infty} \ p(x) \ dx = 1$
	\item Now, we draw samples from this distribution
	\end{itemize}
	}
	%\includemovie{5cm}{2cm}{images/drawingfromdist.gif}
	%\animategraphics[loop, autoplay, width=0.9\textwidth]{1}{images/draw-}{0}{9} % TODO! THIS DOESN'T WORK.
	\only<2>{\includegraphics[width=0.9\textwidth]{images/draw-0}}
	\only<3>{\includegraphics[width=0.9\textwidth]{images/draw-1}}
	\only<4>{\includegraphics[width=0.9\textwidth]{images/draw-2}}
	\only<5>{\includegraphics[width=0.9\textwidth]{images/draw-3}}
	\end{overlayarea}
	\end{block}
\end{frame}

\begin{frame}
  \frametitle{Concept 2: Repetition}
	We have already applied `Concept 2: Drawing from continuous distributions' in both previous ABMs in Lectures 9 and 10: \\ \ \\
	Agents (e.g.\ foxes and rabbits) were randomly spawned on a 2D space $(x, y)$ with $x, y \, \in \, [0, 1]$.
        \newline
        Notation: \textit{Uniform distribution} between 0 and 1.
%	Each allowed value for the random variables $x$ and $y$ (i.e.\ between $0$ and $1$) is equally likely.
	\begin{center}
	\includegraphics[width=0.4\linewidth]{images/uniform_xy}
	\end{center}
	
\end{frame}



\begin{frame}[t]
  \frametitle{Concept 2: Common continuous distributions}
%	\begin{overlayarea}{\linewidth}{12cm}
\only<1>{
	\begin{block}{Uniform distribution}
	\includegraphics[width=0.9\textwidth]{images/UnifDist}
	
	\begin{tabularx}{\textwidth}{L{1.2cm}|X}
		What & continuous, bounded range (max and min are known) \\ \hline
		PDF & $\mathcal{U}(x_{\rm min}, x_{\rm max}) = \frac{1}{x_{\rm max} - x_{\rm min}}$  \\ \hline
		Usage & Uninformative. Use when we have no clue about the random variable.  	
	\end{tabularx}
	\end{block}
	}
\only<2>{
	\begin{block}{Gaussian or Normal distribution}
	\includegraphics[width=0.9\textwidth]{images/GaussDist}
	
	\begin{tabularx}{\textwidth}{L{1.2cm}|X}
What & continuous, infinite range  \\ \hline
PDF & $\mathcal{N}(\mu, \sigma) = \frac{1}{\sqrt{2 \pi \sigma^2}} \cdot \exp \left(-\frac{(x-\mu)^2}{2\sigma^2}\right)$   \\ \hline
          Usage & often observed in nature \ra law of large numbers.  \\
          & \ra very easy to use analytically. \\
	\end{tabularx}
\end{block}
}

\only<3>{
	\begin{block}{Delta distribution}
	\includegraphics[width=0.9\textwidth]{images/DeltaDist}

\begin{tabularx}{\textwidth}{L{1.2cm}|X}
	What & continuous, inifinte/bounded range  \\ \hline
PDF & $\delta(x-\tilde{x}) =: \delta_{\tilde{x}} = \begin{cases}
	\infty \text{ if } x=\tilde{x} \\ 0 \text{ else}
\end{cases}$  \quad $\int \delta(x-\tilde{x}) {\rm d}x := 1$ \\ \hline
Usage & We are absolutely certain about the parameter $x$, e.g.\ $g=9.81\, m/s^2$! Typically, we simply fix $x=\tilde{x}$ 
\end{tabularx}
\end{block}
}

\only<4>{
\begin{block}{Gamma distribution}
\begin{columns}
	\begin{column}{0.35\textwidth}
		\includegraphics[width=1\linewidth]{images/GammaDist} 
		{\scriptsize$\Gamma(5.9807,\ 0.948)$ }
	\end{column}
	\begin{column}{0.6\textwidth}
				\begin{itemize} \item $x$ semi-bounded $[0, \infty]$
				\item often used as distribution "close to Gaussian with long tail" (e.g.\ salary of people) \end{itemize}
	\end{column}
\end{columns}
\end{block}

\begin{block}{Beta distribution}
\begin{columns}
	\begin{column}{0.35\textwidth}
		\includegraphics[width=1\linewidth]{images/BetaDist} 	
				{\scriptsize	${\rm Beta}(1,3)$} 
	\end{column}
	\begin{column}{0.6\textwidth}
          \begin{itemize}
          \item $x$ bounded between $[0,1]$
          \item often used for parameters that represent uncertain probabilities
         \end{itemize}
	\end{column}
\end{columns}
\end{block}

	{\tiny \ra There are soo many distributions
	\url{https://en.wikipedia.org/wiki/List_of_probability_distributions}}
}
\end{frame}

\begin{frame}[fragile]
  \frametitle{Concept 2: Python Package `scipy.stats'}
  \begin{overlayarea}{\textwidth}{\textheight}
   \begin{onlyenv}<1>
\begin{block}{`scipy.stats'}
\begin{itemize} 
\item \lstinline|import scipy.stats as stats|
\item Create distribution e.g.\ via \lstinline|stats.norm(mu, sigma)|
\item 
\item \phantom{Whey}
\end{itemize}
\end{block}
\textbf{Create distribution:}
\begin{lstlisting}[basicstyle=\tiny\ttfamily, tabsize=6]
import scipy.stats as stats

mu = 1.2
sigma = 0.2
some_normal_dist = stats.norm(mu, sigma)
\end{lstlisting}
\begin{itemize} 
\item For other distributions, simply replace `norm' with e.g.\ `beta' and look up what parameters you need to specify!
\item {\scriptsize (as always in python, documentation is your friend \url{https://docs.scipy.org/doc/scipy/reference/stats.html} incl.\ examples and explanations of the parameters to specify, \ldots)}
\end{itemize}
\end{onlyenv}
\begin{onlyenv}<2,3>
\begin{block}{`scipy.stats'}
\begin{itemize} 
\item \lstinline|import scipy.stats as stats|
\item Create distribution e.g.\ via \lstinline|stats.norm(mu, sigma)|
\item PDF via \lstinline|.pdf(x)|
\item \phantom{Whey}
\end{itemize}
\end{block}

%\begin{block}
\textbf{PDF:}
\begin{lstlisting}[basicstyle=\tiny\ttfamily, tabsize=6]
x = np.linspace(0,2)
plt.plot(x, some_normal_dist.pdf(x))
plt.title(f'PDF of a normal distribution with mu={mu}, sigma={sigma}')
\end{lstlisting}
\only<3->{\Put(145,10){\includegraphics[width=0.6\linewidth]{images/gauss}}}
\end{onlyenv}
\begin{onlyenv}<4->
\begin{block}{`scipy.stats'}
\begin{itemize} 
\item \lstinline|import scipy.stats as stats|
\item Create distribution e.g.\ via \lstinline|stats.norm(mu, sigma)|
\item PDF via \lstinline|.pdf(x)|
\item Sampling via \lstinline|.rvs(samplesize)|
\end{itemize}
\end{block}

\textbf{Draw samples from the distribution:}
\begin{lstlisting}[basicstyle=\tiny\ttfamily, tabsize=6]
samples = some_normal_dist.rvs(100)  # Argument = Nr of samples 
plt.hist(samples)
plt.xlabel("x")
plt.ylabel("frequency")
plt.title("Histogram of samples")
plt.show()
\end{lstlisting}

\only<5->{\Put(145,30){\includegraphics[width=0.6\linewidth]{images/gauss_samples}}}

\end{onlyenv}
\end{overlayarea}
\end{frame}

%\begin{frame}[nonumber]
%Back to the Agents and Covid-19!
% \end{frame}

\againframe{codeoverview}

\begin{frame}
\frametitle{Back to the agent}
So far: \textbf{general properties} for an agent. \newline
Now: \textbf{infection-specific properties}.
\pause
{\small 
\begin{enumerate}[<+->]
\item Time of exposure
\item Symptomatic or asymptomatic?
\item How long is the incubation period? (when do symptoms start?)
\item From when to when is the agent infectious?
\item Will the agent die from the infection?
\item How infectious is the agent? (Is the agent a superspreader?)
\end{enumerate}
}
\visible<2->{\includegraphics[height=0.35\textheight]{images/courseInfection_Corona_noData}}%
\end{frame}


\begin{frame}[fragile]
\frametitle{Catch the virus -- Determine course of infection}
\begin{overlayarea}{\linewidth}{12cm}
\begin{onlyenv}<1>
\begin{lstlisting}[basicstyle=\tiny\ttfamily, tabsize=6] 



def catch_virus(ag, t_exposure):
    ag.health_state = "exposed"  
    ag.t_e = t_exposure
\end{lstlisting}
\vspace{0.5cm}
\fbox{\textbf{1}:}\newline the agent \textit{ag} catches the virus (\textit{catch\_virus}) at time \textit{t=t\_exposure}, i.e.\ after being infected with the virus by another agent. \\
First, the \textit{health\_state} of the agent changes to \textit{"exposed"} and the time of infection is saved in the internal variable \textit{t\_e}.
\end{onlyenv}

\begin{onlyenv}<2>
\begin{lstlisting}[basicstyle=\tiny\ttfamily, tabsize=6] 
P_SYMPTOMATIC = {"child": 0.1, "adult": 0.5, "elderly": 0.8} 


def catch_virus(ag, t_exposure):
    ...
    p_s = P_SYMPTOMATIC[ag.age]
    ag.symptomatic = True if np.random.random() < p_s else False
\end{lstlisting}
\vspace{0.5cm}
\fbox{\textbf{2}:}\newline determine whether the infection is
\begin{itemize}
\item symptomatic (\textit{ag.symptomatic = True}) with probabiltiy \textit{p\_s} or
\item asymptomatic (\textit{ag.symptomatic = False})
\end{itemize}
(\,\HandRight \ sample from two options with some probability). \\
The probability depends on age (vulnerability).
\end{onlyenv}

\begin{onlyenv}<3>
\begin{lstlisting}[basicstyle=\tiny\ttfamily, tabsize=6] 
INCUBATION_PERIOD_DIST = stats.gamma(5.807, 0.948)


def catch_virus(ag, t_exposure):
    ...
    incubation_period = INCUBATION_PERIOD.rvs()
\end{lstlisting}
\vspace{0.3cm}
\fbox{\textbf{3}:}\newline determine incubation period by drawing a sample from a gamma distribution inferred from data by \citet{lauer_incubation_2020} \newline (\,\HandRight\ draw from continuous distr.). 
\begin{center}
	\includegraphics[width=0.6\textwidth]{images/GammaDist_}
\end{center}
{\tiny Note: For asymptomatic cases the incubation has no meaning. It's just used as a characteristic value to determine the infectiousness period (next slide).}
\end{onlyenv}

\begin{onlyenv}<4>
\begin{lstlisting}[basicstyle=\tiny\ttfamily, tabsize=6] 



def catch_virus(ag, t_exposure):
    ...
    if ag.symptomatic:
	# Symptomatic Case
	ag.t_onset_symptoms = ag.t_e + incubation_period
    else:
	# Asymptomatic Case
	ag.t_onset_symptoms = np.nan
\end{lstlisting}
\vspace{0.5cm}
\fbox{\textbf{3}:}\newline symptoms start after the incubation period, if \textit{ag}'s infection is symptomatic. If \textit{ag} is asymptomatic, simply ignore.
\end{onlyenv}
\begin{onlyenv}<5>
\begin{lstlisting}[basicstyle=\tiny\ttfamily, tabsize=6] 
TIME_I_PRESYMPT = 2
TIME_I_POSTSYMPT = 4

def catch_virus(ag, t_exposure):
    ...
    ag.infectious_period = [
        ag.t_e + incubation_period - TIME_I_PRESYMPT,
        ag.t_e + incubation_period + TIME_I_POSTSYMPT
    ]
\end{lstlisting}
\vspace{0.5cm}
\fbox{\textbf{4}:}\newline the agent is only infectious two days before and four days after the onset of symptoms (or, for asymptomatic cases, the theoretical onset). 

\begin{center}
  %	\includegraphics[width=0.7\linewidth]{images/single_agents_sickness}
\includegraphics[height=0.2\textheight]{images/courseInfection_Corona_noData}%
\end{center}
\end{onlyenv}

\begin{onlyenv}<6>
\begin{lstlisting}[basicstyle=\tiny\ttfamily, tabsize=6] 
CFR = {"child": 0.00001, "adult": 0.005, "elderly": 0.05} 


def catch_virus(ag, t_exposure):
	...
	if ag.symptomatic:
		# Symptomatic Case, might die
		p_d = CFR[ag.age]
		ag.fatal_outcome = True if np.random.random() < p_d else False
	else:
		# Asymptomatic Case, can not die
		ag.fatal_outcome = False
\end{lstlisting}
  \fbox{\textbf{5}:} \newline
  if symptomatic, agent \textit{ag} might die (\textit{ag.fatal\_outcome = True}). Here, asymptomatic agents do not die.
\\
The probability of dying depends strongly on age. \\
\end{onlyenv}

\begin{onlyenv}<7>
\begin{lstlisting}[basicstyle=\tiny\ttfamily, tabsize=6]
BASE_I = stats.beta(1, 3)


def catch_virus(ag, t_exposure):
	...
	ag.base_infectiousness = BASE_I.rvs() 	# max probability to infect others
\end{lstlisting}
\fbox{\textbf{6}:}\newline each agent has a \textit{ag.base\_infectiousness} ($\in [0,1]$), a scale-factor determining how likely the agent will infect others when they interact. \\
\begin{center}
	\includegraphics[width=0.5\textwidth]{images/BetaDist_} \newline
	{\scriptsize beta distribution: a few agents are highly infectious, most are barely infectious.}
\end{center}
{\footnotesize Later: infectiousness depends on stage/type of infection (a- or pre-symptomatic people tend to be less infectious than symptomatic people) \ra defined later}
%\end{onlyenv}
\end{onlyenv}

\begin{onlyenv}<8>
Summary
\begin{lstlisting}[basicstyle=\tiny\ttfamily, tabsize=6]
def catch_virus(ag, t_exposure):
    ag.health_state = "exposed"  
    ag.t_e = t_exposure

    p_s = P_SYMPTOMATIC[ag.group]
    ag.symptomatic = True if np.random.random() < p_s else False

    incubation_period = INCUBATION_PERIOD.rvs()
    
    ag.infectious_period = [
        ag.t_e + incubation_period - TIME_I_PRESYMPT,
        ag.t_e + incubation_period + TIME_I_POSTSYMPT
    ]
    
    if ag.symptomatic:
        ag.t_onset_symptoms = ag.t_e + incubation_period
        p_d = CFR[ag.group]
        ag.fatal_outcome = True if np.random.random() < p_d else False
    else:
        ag.t_onset_symptoms = np.nan
        ag.fatal_outcome = False
        
    ag.base_infectiousness = BASE_I.rvs()   # * FACTOR_INFECTIOUSNESS
	
	return 
\end{lstlisting}
\end{onlyenv}
\end{overlayarea}
\end{frame}




\begin{frame}
\frametitle{A few examples of infection courses [Assignment]}
\begin{enumerate}
	\item Create an agent
	\item The agent catches the virus (i.e.\ call \textit{catch\_virus})
	\item Plot the course of the infection. 
	\item Repeat 1-3
\end{enumerate}

\pause
\begin{center}
	\includegraphics[width=0.9\linewidth]{images/agents_sickness}
\end{center}
\end{frame}


%%% Local Variables:
%%% mode: latex
%%% TeX-master: "CoronaABM"
%%% End:


\begin{frame}[fragile]
\frametitle{\textbf{Concept 1 and 2 Summary}}
\begin{block}{Drawing from distributions}
\begin{itemize}
	\item We can create heterogeneous agents / courses of infections /... by drawing new parameters or properties from probability distributions when we initialise the agent/determine the course of the infection/ ...
	\item For discrete choices: 
\begin{lstlisting}[basicstyle=\tiny\ttfamily, tabsize=6] 
for ag in range(N_AGENTS):
	ag = Agent()
	ag.property1 = np.random.choice(all_choices, p = probs_for_choices)
\end{lstlisting}
	\item For continuous random variables (here normally distributed):
\begin{lstlisting}[basicstyle=\tiny\ttfamily, tabsize=6] 
import scipy.stats as stats
for ag in range(N_AGENTS):
	ag = Agent()
	ag.property2 = stats.norm(mu, sigma).rvs()
	# distributed according to stats.norm(mu, sigma).pdf(x)
\end{lstlisting}
\end{itemize}
\end{block}
Repetition: When do we want to exploit this?

\end{frame}

\againframe<4>{overview}

%\frame{BREAK}

\againframe<3>{Outline}

\section{Concept 3: Networks}
%\begin{frame}{Social Networks}
%
%Some slides from Ago:
%\bi 
%\item What is a network
%\item What is the meaning of nodes, links
%\item Mention directed/undirected/weighted
%\item Different topologies (mainly small-world and scale-free)
%\bi 
%\item Node Degree, 
%\item Clustering coeff
%\ei
%\item The networkx package
%\ei
%\end{frame}


\begin{frame}{Social Network -- Basics}
	With which agents do agents interact? I.e.\ who will be possibly infected by a infectious agent? \ra Social Network
	\bi
	\item Nodes: \quad Each node represents one agent
	\item Link/Edge between nodes: \quad  Agents are in `physical contact'
	
	\only<1>{\centering \includegraphics[width=0.4\linewidth]{WattsStrogatz/network}}
	\item<2-> (averageg) Node degree = (avg) Nr of links from the agent
	\item<3-> Adjacency Matrix and List: Alternative network representation. \newline
			Entry in matrix is 1 = "there is a link between the nodes with index corresponding to row and column of the entry".
	\ei
	\centering
\only<1>{\includegraphics[width=0.4\linewidth]{WattsStrogatz/network}}	\only<3->{\includegraphics[width=0.7\linewidth]{Networks/images/adj}}
\end{frame}



\begin{frame}{Social Network - Topology}
\begin{overlayarea}{\linewidth}{12cm}
\bi 
\item We will use one particular network: A `Watts-Strogatz network', often also referred to as `small-world network'.
\bi 
\item<2-> All $n$ nodes/agents are aligned in a ring.
\item<3-> They are connected to their $k$ nearest neighbours (left/right)
\item<4-> With probability $p$, each link is capped and re-drawn to a random node/agent anywhere on the ring.
\ei 
\ei
\begin{center}
	\only<1>{\vspace{3cm}}
	
	\only<2>{\includegraphics[width=0.35\linewidth]{images/watts1}}
	
	\only<3>{\includegraphics[width=0.35\linewidth]{images/watts2}}
	
	\only<4>{\includegraphics[width=0.35\linewidth]{images/watts3}}
	
	\only<5>{\includegraphics[width=0.7\linewidth]{images/watts_strogatz}}
\end{center}
{\tiny  D. J. Watts \& S. H. Strogatz, Collective dynamics of 'small-world'
	networks, \textit{Nature}, 393:440--442, 1998.}	
\end{overlayarea}
\end{frame}

\begin{frame}{Social Network -- Perspective}
	\bi 
	\item Network Theory is one of the `hottest' topics in complexity science (e.g.\ neural networks)
	\item The method can be applied on various systems, topics, problems 
	\item Some extensions: 
	\bi 	
	\item Directed and weighted links
	\item Topologies of different networks:
		\bi 
		\item Scale-free network \ra e.g.\ Barabási-Albert-Modell
		\item Random Graph
		\ei
	\item Clustering, Clustering Coefficient.
	\item Co-evolving networks, i.e.\ that change over time depending on the state of the system.
	\ei
	\ei
\end{frame}

\begin{frame}[fragile]
\frametitle{The \textit{Networkx} Package in Python}

\begin{overlayarea}{\linewidth}{12cm}
\begin{onlyenv}<1->
\begin{lstlisting}[basicstyle=\tiny\ttfamily, tabsize=6] 
import networkx as nx
G = nx.Graph()
G.add_node("Ahmad")
...
G.add_edge("Ahmad", "Can")
...
pos = nx.spring_layout(G)	# Just a `nice' way of arranging the nodes
nx.draw(G, pos, with_labels=True)
\end{lstlisting}
\end{onlyenv}
\vspace{1cm}

\centering
\begin{columns}
	\begin{column}{0.4\textwidth}
		\only<1-5>{\includegraphics[width=0.7\linewidth]{WattsStrogatz/network}}	
	\end{column}
	\begin{column}{0.6\textwidth}
	
	\only<2-5>{What's the adjacency Matrix?}
	\only<2>{
		\phantom{$\begin{pmatrix} Deborah\ D\end{pmatrix} \Rightarrow $\ \ }
		$\begin{pmatrix} A & B & C & D \end{pmatrix} $ \\
		\phantom{$\begin{pmatrix} Deborah\ D\end{pmatrix} \Rightarrow \begin{pmatrix} A & B \end{pmatrix}$} $\Downarrow$ \\
		$\begin{pmatrix} Ahmad\ A\\ Beate\ B\\ Can\ C\\Deborah\ D\end{pmatrix} \Rightarrow
		\begin{pmatrix}
			\color{white}{0}& \color{white}{1}& \color{white}{1}& \color{white}{1}\\
			& & & \\
			& & & \\
			& & & 
		\end{pmatrix}$
	}
		
	\only<3>{
		\phantom{$\begin{pmatrix} Deborah\ D\end{pmatrix} \Rightarrow $\ \ }
		$\begin{pmatrix} A & B & C & D \end{pmatrix} $ \\
		\phantom{$\begin{pmatrix} Deborah\ D\end{pmatrix} \Rightarrow \begin{pmatrix} A & B \end{pmatrix}$} $\Downarrow$ \\
		$\begin{pmatrix} Ahmad\ A\\ Beate\ B\\ Can\ C\\Deborah\ D\end{pmatrix} \Rightarrow
			\begin{pmatrix}
			\color{gray}{0}& 1& 1& 1\\
			& & & \\
			& & & \\
			& & & 
		\end{pmatrix}$
	}
	\only<4>{
		\phantom{$\begin{pmatrix} Deborah\ D\end{pmatrix} \Rightarrow $\ \ }
		$\begin{pmatrix} A & B & C & D \end{pmatrix} $ \\
		\phantom{$\begin{pmatrix} Deborah\ D\end{pmatrix} \Rightarrow \begin{pmatrix} A & B \end{pmatrix}$} $\Downarrow$ \\
		$\begin{pmatrix} Ahmad\ A\\ Beate\ B\\ Can\ C\\Deborah\ D\end{pmatrix} \Rightarrow
		\begin{pmatrix}
			\color{gray}{0}& 1& 1& 1\\
			\color{gray}{1}&\color{gray}{0}& 0& 0\\
			\color{gray}{1}& \color{gray}{0}& & \\
			\color{gray}{1}& \color{gray}{0}& & 
		\end{pmatrix}$
	}
	\only<5>{
		\phantom{$\begin{pmatrix} Deborah\ D\end{pmatrix} \Rightarrow $\ \ }
		$\begin{pmatrix} A & B & C & D \end{pmatrix} $ \\
		\phantom{$\begin{pmatrix} Deborah\ D\end{pmatrix} \Rightarrow \begin{pmatrix} A & B \end{pmatrix}$} $\Downarrow$ \\
		$\begin{pmatrix} Ahmad\ A\\ Beate\ B\\ Can\ C\\Deborah\ D\end{pmatrix} \Rightarrow
		\begin{pmatrix}
			\color{gray}{0}& 1& 1& 1\\
			\color{gray}{1}& \color{gray}{0}& 0& 0\\
			\color{gray}{1}& \color{gray}{0}& \color{gray}{0}& 1\\
			\color{gray}{1}& \color{gray}{0}& \color{gray}{1} & \color{gray}{0}
		\end{pmatrix}$
	}
\end{column}
\end{columns}

\begin{onlyenv}<6->
Building a Watts-Strogatz Network is even easier:
\begin{lstlisting}[basicstyle=\tiny\ttfamily, tabsize=4] 
WS = nx.watts_strogatz_graph(
					N_AGENTS,   	# n = How many nodes
					4, 	  	# k = How many nearest neighbours
					0.1)	# p = Probability for each link to be rewired
		
pos = nx.spring_layout(WS)
nx.draw(WS, pos, with_labels=True)

print("The adjacency matrix is: ", nx.adjacency_matrix(WS))
print("The adjacency list for agent `ag' is: ", WS.adj[ag.id])
	\end{lstlisting}
\end{onlyenv}

		\end{overlayarea}
\end{frame}

\begin{frame}[fragile]
\frametitle{\textbf{Concept 3 Summary}}
\begin{block}{Social Network}
\bi 
\item Social networks can be used to represent communication/interaction between agents
\item Topology of the network matters (in particular w.r.t.\ clustering and node degree)
\item The package networkx is wonderful and simple to use:
\begin{lstlisting}[basicstyle=\tiny\ttfamily, tabsize=6]
import networkx as nx
G = nx.watts_strogatz_graph(100, 4, 0.1)
nx.draw(G)
print("Agents have contacts to the nodes/agents with these indices: ", G.adj)
\end{lstlisting}
\ei
\end{block}
\end{frame}

\againframe<5>{overview}

%!Tex root = CoronaABM.tex


%\begin{frame}{Social Network in Covid-19 ABM}
%
%General idea
%
%\end{frame}


%\begin{frame}[fragile]
%	\frametitle{Code}
%	
%
%
%\end{frame}





  \section{Update Function and Run Function}

\againframe{codeoverview}

\begin{frame}[fragile]
	\frametitle{Update Function}
	\begin{itemize} 
	\item<1-> Choose queuing order of agents (\textit{np.random.choice}).
	\item<2-> For each agent:
	\begin{itemize} 
	\item<2-> Check (and update) health state \ra state-dependent action  
	\item<2-> Potentially infect others in network with certain probability
	\end{itemize}
      \end{itemize}
	\begin{overlayarea}{\linewidth}{100cm}
\begin{onlyenv}<1,2>
  \begin{lstlisting}[basicstyle=\tiny\ttfamily, tabsize=6]

    
def update(t_now):
	queue = np.random.choice(agents, size=N_AGENTS, replace=False)
	for ag in queue:
\end{lstlisting}
\end{onlyenv}

\begin{onlyenv}<3>
  \begin{lstlisting}[basicstyle=\tiny\ttfamily, tabsize=6]

    
def update(t_now):
	queue = np.random.choice(agents, size=N_AGENTS, replace=False)
	for ag in queue:
		if ag.health_state == "susceptible":
			pass  # Do nothing
\end{lstlisting}
\end{onlyenv}

\begin{onlyenv}<4>
  \begin{lstlisting}[basicstyle=\tiny\ttfamily, tabsize=6]

    
def update(t_now):
	queue = np.random.choice(agents, size=N_AGENTS, replace=False)
	for ag in queue:
		if ag.health_state == "susceptible":
			pass  # Do nothing
		if ag.health_state == "exposed":
			# Potentially become infectious
			if t_now >= ag.infectious_period[0]:
				if ag.symptomatic:
					ag.health_state = "infectious_presymptomatic" 
				else:
					ag.health_state = "infectious_asymptomatic"
\end{lstlisting}
Switch from latent to infectious \textit{health\_state} (pre-symptom or asymptomatic) when \textit{infectious\_period} starts
\end{onlyenv}
\begin{onlyenv}<5>
  \begin{lstlisting}[basicstyle=\tiny\ttfamily, tabsize=6]

    
def update(t_now):
	queue = np.random.choice(agents, size=N_AGENTS, replace=False)
	for ag in queue:
		if ag.health_state == "susceptible":
			...
		if ag.health_state == "exposed":
			...
		if ag.health_state == "infectious_presymptomatic":  
			if t_now >= ag.t_onset_symptoms:
				ag.health_state = "infectious_symptomatic"
\end{lstlisting}
Switch from pre-symptomatic to symptomatic when incubation period is over.
\end{onlyenv}

\begin{onlyenv}<6>
  \begin{lstlisting}[basicstyle=\tiny\ttfamily, tabsize=6]

    
def update(t_now):
	queue = np.random.choice(agents, size=N_AGENTS, replace=False)
	for ag in queue:
		if ag.health_state == "susceptible":
			...
		if ag.health_state == "exposed":
			...
		if ag.health_state == "infectious_presymptomatic":  
			...
		if "infectious" in ag.health_state:  
			if t_now >= ag.infectious_period[1]:
				if ag.fatal_case:
					ag.health_state = "dead"
				else:
					ag.health_state = "recovered" 
\end{lstlisting}
If agent in \textit{health\_state ="infectious\_..."} and the infectious period is over, then either recover or die.
\end{onlyenv}

\begin{onlyenv}<7>
\begin{lstlisting}[basicstyle=\tiny\ttfamily, tabsize=6] 

  
def update(t_now):
	queue = np.random.choice(agents, size=N_AGENTS, replace=False)
	for ag in queue:
		...
		if "infectious" in ag.health_state:
                    infect_others(ag, t_now)
\end{lstlisting}

                      \vspace{0.2cm}
\begin{lstlisting}[basicstyle=\tiny\ttfamily, tabsize=6] 
RELATIVE_INFECTIOUSNESS = {"infectious_presymptomatic": 0.5, "infectious_symptomatic": 1, "infectious_asymptomatic": 0.2}

def infect_others(ag, t):
      p_i = ag.base_infectiousness * RELATIVE_INFECTIOUSNESS[ag.health_state]	
      # Loop through contacts and potentially infect them
      linked_contacts = list(network.adj[ag.id])  # Indices of neighbours
      for c in linked_contacts:
            contact_person = agents[c]
            if contact_person.health_state == "susceptible"
                  if np.random.random() < p_i:
                        catch_virus(contact_person, t)
\end{lstlisting}
              \end{onlyenv}
\end{overlayarea}
\end{frame}

%%% Local Variables:
%%% mode: latex
%%% TeX-master: "CoronaABM"
%%% TeX-master: "CoronaABM"
%%% TeX-master: "CoronaABM"
%%% End:


% !TeX root = CoronaABM.tex

%\begin{frame}
%	\frametitle{Observe Function}
%	Don't show code in detail, just results \ra next slides.\\
%	Show how to get these aggregate results (list comprehension and ask for state, then count states)
%\end{frame}

\begin{frame}[fragile]
	\frametitle{Run and Observe Function}

Goal: Perform simulation and track how many agents are in state "susceptible", "exposed", ... over time.
\begin{overlayarea}{\linewidth}{12cm}
\begin{onlyenv}<1>
\begin{lstlisting}[basicstyle=\tiny\ttfamily, tabsize=6] 
initialise()  # Initialise all agents
Net = initialise_network(agents, k=K_WS_NETWORK, p=P_WS_NETWORK)

T_ARRAY = np.linspace(0, 100, 0.5)

states = ["susceptible", "exposed", "inf_asympt", "inf_presympt", "inf_sympt", "recovered", "dead"]
results = np.empty([len(T_ARRAY), len(states)])

for n, t in enumerate(T_ARRAY):
	update(t)
	
	states_of_agents = [ag.state for ag in agents]
	N_s = states_of_agents.count("susceptible")
	N_e = states_of_agents.count("exposed")
	...
	results[n, :] = np.array([N_s, N_e, N_ia, ...])
\end{lstlisting}
\end{onlyenv}
\begin{onlyenv}<2>
\begin{figure}
	\centering
	\includegraphics[width=0.8\linewidth]{AggregateResults_N-1000-2_WS(6,2e-01)_seed2}
	\caption{An outbreak in a network with k=6, p = 0.2, 1000 agents of which 2 are exposed at t=0.}
	\label{fig:aggregateresultsn-1000-2ws62e-01seed1}
\end{figure}
\end{onlyenv}
\begin{onlyenv}<3>
	\begin{figure}
		\centering
		\includegraphics[width=0.8\linewidth]{AggregateResults_N-1000-2_WS(6,2e-01)_seed3}
		\caption{A \textit{different} outbreak in a network with k=6, p = 0.2, 1000 agents of which 2 are exposed at t=0.}
		\label{fig:aggregateresultsn-1000-2ws62e-01seed1}
	\end{figure}
\end{onlyenv}
\begin{onlyenv}<4>
\begin{figure}
	\centering
	\includegraphics[width=0.8\linewidth]{AggregateResults_N-1000-2_WS(6,2e-01)_seed1}
	\caption{This simulation is {\color{red}stochastic}! Even though the same model setup is used it might not lead to an outbreak (a qualitatively difference!). Model output depends on the actual course of microscopic events.}
\end{figure}
 %This simulation is {\color{red}stochastic}! Model output depends on the actual course of microscopic events.
\end{onlyenv}
\end{overlayarea}
\end{frame}

\againframe<6>{overview}

\begin{frame}
	\frametitle{Validation}
How can we verify if these results make sense? E.g.\ 
\bi 
\item Ensemble Runs: Do many different simulation runs.
\item Observe aggregate indicators
\bi 
\item The initial reproductive number$R_0$. 
\item Serial Interval (not done here)
\ei
\ei
\pause%\begin{onlyenv}<2>
\includegraphics[width=0.7\linewidth]{R0values_seed10}
%\end{onlyenv}

\end{frame}

\section{Policies}
\begin{frame}{Policies}
	The major idea of such a model was to test the impact of local policies. \\
	E.g.\ What happens 
	\bi 
	\item if the number of contacts is reduced (due to a socially distancing public)?
	\item or if the people keep contacts within tight clusters (households, neighbours)?
	\item or if the agents reduce their $base\_infectiousness$, by wearing a mask?	
	\ei 
	
\end{frame}

\begin{frame}{Parameters of the Model}
\begin{table}
	\scriptsize
\begin{tabularx}{\textwidth}{>{\tiny}l|X|p{2cm}}
	Parameter & Description & current value \\ \hline
	FRAC\_RISKGROUPS & Percentage of each risk group & [0.2, 0.5, 0.3]  \\
	BASE\_I & Distribution of base infectiousness of the agents & beta(1, 3) \\
	P\_SYMPTOMATIC & Probability to develop symptoms  & [0.1, 0.5, 0.8] \\
	INCUBATION\_PERIOD & Distribution of the incubation time & gamma(5.807, 0.948)\\
	CFR & Case fatality ratio for each age group & [0.0001, 0.005, 0.05] \\
	I\_SYMPTOMATIC & Strength of infectiousness for a symptomatic infection & 1 \\ 
	I\_ASYMPTOMATIC & -"- for an asymptomatic infection &  0.2 \\  
	I\_PRESYMPTOMATIC& -"- for a pre-symptomatic infection &  0.5 \\   
	TIME\_I\_PRESYMPT & infectious days before symptom onset & 2  \\
	TIME\_I\_POSTSYMPT & infectious days after symptom onset & 4  \\
	K\_WS\_NETWORK & Number of (nearest) neighbours in networks for nodes &  6 \\
	P\_WS\_NETWORK & Probability of each link to be rewired &  0.2  \\
	N\_AGENTS & Nr of agents & 1000 \\ %  # Number of agents in total
	N\_INFECTED\_INIT & Nr of agents set to "exposed" at t=0 & 2  \\
\end{tabularx}
\end{table}

\end{frame}

\begin{frame}{Policy: Decrease $k$ of Network to $k=4$}
\begin{figure}
	\only<1>{\includegraphics[width=\linewidth]{../../../CoronaABM/Simplified_Corona/Figures/AggregateResults_N-1000-2_WS(4,2e-01)_seed2}}
	\only<2>{
			\includegraphics[width=\linewidth]{../../../CoronaABM/Simplified_Corona/Figures/AggregateResults_N-1000-2_WS(4,2e-01)_seed2}
			\Put(-40,80){\includegraphics[width=0.7\linewidth]{../../../CoronaABM/Simplified_Corona/Figures/R0values_seed2}}}
	\end{figure}
\end{frame}



%\againframe<6>{overview}

\begin{frame}{What we have COVERED today.}
	\begin{itemize}
		\item An Agent-Based Model of the spread of Covid-19 in a small society in order to test local policies
		\item How to make agents heterogeneous: Drawing from probability distributions
		\item How to let agents interact with each other: Social Networks for the interaction of agents
	\end{itemize}
\end{frame}

\begin{frame}{Assignment}
\only<1>{\bi
	\item Basic:
	\bi 
	\item Read and understand the code!
	\item Run it with several different \textit{seed} values
	\item Change the network topology and properties (e.g.\ \textit{k} or \textit{p} of the Watts-Strogatz Network). What policy could this correspond to?
	\item Try an entirely different network (\ra networkx documentation). 
	\item Take the distributions for \textit{incubation period} or \textit{base infectiousness}, draw samples from them and plot their PDF and the histogram of frequencies (see Concept 2). {\scriptsize Note: You may want to create a separate script for this.}
	\item Select one of these distributions in the Covid-19 Model and change it (e.g.\ decrease the incubation period or make all agents equally infectious). Find something that interests you!
	\ei 
	\ei 
}
\only<2>{\bi
	\item Intermediate:
	\bi
	%\item Imagine you want to implement how wearing a face mask changes the results, which parameter/process would you change? Test this.
	\item Implement a soft isolation policy: I.e.\ when an agent turns symptomatic, she/he will quarantine and strongly reduce further contacts. \newline
	{\scriptsize Hint: For this you might want to add a line in the \textit{update} function that reduces the number of contacts when an isolated, symptomatic agent is about to infect these.}
	\item This policy may not be in place immediately, but only some time after the outbreak has been noticed. Implement a delay of the policy implementation. How does such a delay impact the effectiveness of a policy?
	\ei 
	\ei
}
\only<3>{
	\bi
	\item Hard, if you feel like exploring:
	\bi 
	\item Implement your own (time-dependent) policy strategy
	\bi 
	\item This could include dynamic changes in the network
	\item Or different behaviour of each age-group.
	\ei
	\item Let a few agents be defectors that do not adhere to the policies.
	\ei 
	\ei
	\vfill 
	Note: Whenever you program anything, make sure you test (for yourself), e.g.\ by varying the crucial parameters, whether the results are what you expect. \\
}
\vfill 
Deadline: Wednesday, {\color{red} 2nd December 2020}, send relevant code, a PDF with your conclusions (including figures) to \href{mailto:peter.steiglechner@leibniz-zmt.de}{peter.steiglechner@leibniz-zmt.de}. As always, please engage if you have any questions or ideas that you want to discuss. 
\end{frame}

\begin{frame}{References}
	\renewcommand*{\bibfont}{\tiny} 
\printbibliography
\end{frame}
\begin{frame}
	\frametitle{Further reading}
	\begin{thebibliography}{10}
		\tiny
		\beamertemplatebookbibitems % block for books
		\bibitem[Sayama, 2015]{SAYA15}
		Hiroki Sayama~2015
		\newblock {\em Introduction to the Modeling and Analysis of
			Complex Systems}
		\newblock Open SUNY Textbooks 

		
		\beamertemplatearticlebibitems % block for articles
		
		\beamertemplateonlinebibitems % block for online material
		\bibitem[NetworkX 2.0, 2017]{NX2.0}
		NetworkX Reference 2.0
		\newblock \url{https://networkx.github.io/documentation/stable/_downloads/networkx_reference.pdf}
		\newblock Sep 20, 2017	
		
	\end{thebibliography}
	
\end{frame}


\end{document}
